% Options for packages loaded elsewhere
\PassOptionsToPackage{unicode}{hyperref}
\PassOptionsToPackage{hyphens}{url}
%
\documentclass[
  11pt,
  b5paper,
  nottoc]{book}

\usepackage{amsmath,amssymb}
\usepackage{lmodern}
\usepackage{iftex}
\ifPDFTeX
  \usepackage[T1]{fontenc}
  \usepackage[utf8]{inputenc}
  \usepackage{textcomp} % provide euro and other symbols
\else % if luatex or xetex
  \usepackage{unicode-math}
  \defaultfontfeatures{Scale=MatchLowercase}
  \defaultfontfeatures[\rmfamily]{Ligatures=TeX,Scale=1}
  \setmonofont[Scale=0.7]{Source Code Pro}
\fi
% Use upquote if available, for straight quotes in verbatim environments
\IfFileExists{upquote.sty}{\usepackage{upquote}}{}
\IfFileExists{microtype.sty}{% use microtype if available
  \usepackage[]{microtype}
  \UseMicrotypeSet[protrusion]{basicmath} % disable protrusion for tt fonts
}{}
\makeatletter
\@ifundefined{KOMAClassName}{% if non-KOMA class
  \IfFileExists{parskip.sty}{%
    \usepackage{parskip}
  }{% else
    \setlength{\parindent}{0pt}
    \setlength{\parskip}{6pt plus 2pt minus 1pt}}
}{% if KOMA class
  \KOMAoptions{parskip=half}}
\makeatother
\usepackage{xcolor}
\usepackage[left=2cm, right=2cm, top=2.5cm, bottom=2.5cm]{geometry}
\setlength{\emergencystretch}{3em} % prevent overfull lines
\setcounter{secnumdepth}{5}
% Make \paragraph and \subparagraph free-standing
\ifx\paragraph\undefined\else
  \let\oldparagraph\paragraph
  \renewcommand{\paragraph}[1]{\oldparagraph{#1}\mbox{}}
\fi
\ifx\subparagraph\undefined\else
  \let\oldsubparagraph\subparagraph
  \renewcommand{\subparagraph}[1]{\oldsubparagraph{#1}\mbox{}}
\fi

\usepackage{color}
\usepackage{fancyvrb}
\newcommand{\VerbBar}{|}
\newcommand{\VERB}{\Verb[commandchars=\\\{\}]}
\DefineVerbatimEnvironment{Highlighting}{Verbatim}{commandchars=\\\{\}}
% Add ',fontsize=\small' for more characters per line
\usepackage{framed}
\definecolor{shadecolor}{RGB}{241,243,245}
\newenvironment{Shaded}{\begin{snugshade}}{\end{snugshade}}
\newcommand{\AlertTok}[1]{\textcolor[rgb]{0.68,0.00,0.00}{#1}}
\newcommand{\AnnotationTok}[1]{\textcolor[rgb]{0.37,0.37,0.37}{#1}}
\newcommand{\AttributeTok}[1]{\textcolor[rgb]{0.40,0.45,0.13}{#1}}
\newcommand{\BaseNTok}[1]{\textcolor[rgb]{0.68,0.00,0.00}{#1}}
\newcommand{\BuiltInTok}[1]{\textcolor[rgb]{0.00,0.23,0.31}{#1}}
\newcommand{\CharTok}[1]{\textcolor[rgb]{0.13,0.47,0.30}{#1}}
\newcommand{\CommentTok}[1]{\textcolor[rgb]{0.37,0.37,0.37}{#1}}
\newcommand{\CommentVarTok}[1]{\textcolor[rgb]{0.37,0.37,0.37}{\textit{#1}}}
\newcommand{\ConstantTok}[1]{\textcolor[rgb]{0.56,0.35,0.01}{#1}}
\newcommand{\ControlFlowTok}[1]{\textcolor[rgb]{0.00,0.23,0.31}{#1}}
\newcommand{\DataTypeTok}[1]{\textcolor[rgb]{0.68,0.00,0.00}{#1}}
\newcommand{\DecValTok}[1]{\textcolor[rgb]{0.68,0.00,0.00}{#1}}
\newcommand{\DocumentationTok}[1]{\textcolor[rgb]{0.37,0.37,0.37}{\textit{#1}}}
\newcommand{\ErrorTok}[1]{\textcolor[rgb]{0.68,0.00,0.00}{#1}}
\newcommand{\ExtensionTok}[1]{\textcolor[rgb]{0.00,0.23,0.31}{#1}}
\newcommand{\FloatTok}[1]{\textcolor[rgb]{0.68,0.00,0.00}{#1}}
\newcommand{\FunctionTok}[1]{\textcolor[rgb]{0.28,0.35,0.67}{#1}}
\newcommand{\ImportTok}[1]{\textcolor[rgb]{0.00,0.46,0.62}{#1}}
\newcommand{\InformationTok}[1]{\textcolor[rgb]{0.37,0.37,0.37}{#1}}
\newcommand{\KeywordTok}[1]{\textcolor[rgb]{0.00,0.23,0.31}{#1}}
\newcommand{\NormalTok}[1]{\textcolor[rgb]{0.00,0.23,0.31}{#1}}
\newcommand{\OperatorTok}[1]{\textcolor[rgb]{0.37,0.37,0.37}{#1}}
\newcommand{\OtherTok}[1]{\textcolor[rgb]{0.00,0.23,0.31}{#1}}
\newcommand{\PreprocessorTok}[1]{\textcolor[rgb]{0.68,0.00,0.00}{#1}}
\newcommand{\RegionMarkerTok}[1]{\textcolor[rgb]{0.00,0.23,0.31}{#1}}
\newcommand{\SpecialCharTok}[1]{\textcolor[rgb]{0.37,0.37,0.37}{#1}}
\newcommand{\SpecialStringTok}[1]{\textcolor[rgb]{0.13,0.47,0.30}{#1}}
\newcommand{\StringTok}[1]{\textcolor[rgb]{0.13,0.47,0.30}{#1}}
\newcommand{\VariableTok}[1]{\textcolor[rgb]{0.07,0.07,0.07}{#1}}
\newcommand{\VerbatimStringTok}[1]{\textcolor[rgb]{0.13,0.47,0.30}{#1}}
\newcommand{\WarningTok}[1]{\textcolor[rgb]{0.37,0.37,0.37}{\textit{#1}}}

\providecommand{\tightlist}{%
  \setlength{\itemsep}{0pt}\setlength{\parskip}{0pt}}\usepackage{longtable,booktabs,array}
\usepackage{calc} % for calculating minipage widths
% Correct order of tables after \paragraph or \subparagraph
\usepackage{etoolbox}
\makeatletter
\patchcmd\longtable{\par}{\if@noskipsec\mbox{}\fi\par}{}{}
\makeatother
% Allow footnotes in longtable head/foot
\IfFileExists{footnotehyper.sty}{\usepackage{footnotehyper}}{\usepackage{footnote}}
\makesavenoteenv{longtable}
\usepackage{graphicx}
\makeatletter
\def\maxwidth{\ifdim\Gin@nat@width>\linewidth\linewidth\else\Gin@nat@width\fi}
\def\maxheight{\ifdim\Gin@nat@height>\textheight\textheight\else\Gin@nat@height\fi}
\makeatother
% Scale images if necessary, so that they will not overflow the page
% margins by default, and it is still possible to overwrite the defaults
% using explicit options in \includegraphics[width, height, ...]{}
\setkeys{Gin}{width=\maxwidth,height=\maxheight,keepaspectratio}
% Set default figure placement to htbp
\makeatletter
\def\fps@figure{htbp}
\makeatother
\newlength{\cslhangindent}
\setlength{\cslhangindent}{1.5em}
\newlength{\csllabelwidth}
\setlength{\csllabelwidth}{3em}
\newlength{\cslentryspacingunit} % times entry-spacing
\setlength{\cslentryspacingunit}{\parskip}
\newenvironment{CSLReferences}[2] % #1 hanging-ident, #2 entry spacing
 {% don't indent paragraphs
  \setlength{\parindent}{0pt}
  % turn on hanging indent if param 1 is 1
  \ifodd #1
  \let\oldpar\par
  \def\par{\hangindent=\cslhangindent\oldpar}
  \fi
  % set entry spacing
  \setlength{\parskip}{#2\cslentryspacingunit}
 }%
 {}
\usepackage{calc}
\newcommand{\CSLBlock}[1]{#1\hfill\break}
\newcommand{\CSLLeftMargin}[1]{\parbox[t]{\csllabelwidth}{#1}}
\newcommand{\CSLRightInline}[1]{\parbox[t]{\linewidth - \csllabelwidth}{#1}\break}
\newcommand{\CSLIndent}[1]{\hspace{\cslhangindent}#1}

\usepackage{booktabs}
\usepackage{multirow}
\usepackage{array,graphicx}
\usepackage{wrapfig}
\usepackage[normalem]{ulem}
\usepackage{lscape}
\usepackage{rotating}
\useunder{\uline}{\ul}{}
\usepackage{float}
\floatplacement{figure}{H}
\renewcommand{\chaptername}{Capitolul}
\renewcommand{\contentsname}{Cuprins}
\renewcommand\listfigurename{Lista figurilor}
\renewcommand\listtablename{Lista tabelelor}
\renewcommand\figurename{Figura}
\renewcommand\tablename{Tabel}
\renewcommand\bibname{Bibliografie}
\newcommand*\rot{\rotatebox{90}}
\usepackage{tocbibind}
\makeatletter
\makeatother
\makeatletter
\@ifpackageloaded{bookmark}{}{\usepackage{bookmark}}
\makeatother
\makeatletter
\@ifpackageloaded{caption}{}{\usepackage{caption}}
\AtBeginDocument{%
\ifdefined\contentsname
  \renewcommand*\contentsname{Table of contents}
\else
  \newcommand\contentsname{Table of contents}
\fi
\ifdefined\listfigurename
  \renewcommand*\listfigurename{Lista figurilor}
\else
  \newcommand\listfigurename{Lista figurilor}
\fi
\ifdefined\listtablename
  \renewcommand*\listtablename{Lista tabelelor}
\else
  \newcommand\listtablename{Lista tabelelor}
\fi
\ifdefined\figurename
  \renewcommand*\figurename{Figura}
\else
  \newcommand\figurename{Figura}
\fi
\ifdefined\tablename
  \renewcommand*\tablename{Tabel}
\else
  \newcommand\tablename{Tabel}
\fi
}
\@ifpackageloaded{float}{}{\usepackage{float}}
\floatstyle{ruled}
\@ifundefined{c@chapter}{\newfloat{codelisting}{h}{lop}}{\newfloat{codelisting}{h}{lop}[chapter]}
\floatname{codelisting}{Listing}
\newcommand*\listoflistings{\listof{codelisting}{List of Listings}}
\makeatother
\makeatletter
\@ifpackageloaded{caption}{}{\usepackage{caption}}
\@ifpackageloaded{subcaption}{}{\usepackage{subcaption}}
\makeatother
\makeatletter
\@ifpackageloaded{tcolorbox}{}{\usepackage[many]{tcolorbox}}
\makeatother
\makeatletter
\@ifundefined{shadecolor}{\definecolor{shadecolor}{rgb}{.97, .97, .97}}
\makeatother
\makeatletter
\makeatother
\ifLuaTeX
  \usepackage{selnolig}  % disable illegal ligatures
\fi
\IfFileExists{bookmark.sty}{\usepackage{bookmark}}{\usepackage{hyperref}}
\IfFileExists{xurl.sty}{\usepackage{xurl}}{} % add URL line breaks if available
\urlstyle{same} % disable monospaced font for URLs
\hypersetup{
  pdftitle={Introducere în analiza datelor economico-financiare},
  pdfauthor={Ciprian Alexandru-Caragea},
  hidelinks,
  pdfcreator={LaTeX via pandoc}}

\title{Introducere în analiza datelor economico-financiare}
\usepackage{etoolbox}
\makeatletter
\providecommand{\subtitle}[1]{% add subtitle to \maketitle
  \apptocmd{\@title}{\par {\large #1 \par}}{}{}
}
\makeatother
\subtitle{Aplicații și exemple rezolvate prin: R, Python, Excel,
PowerBI}
\author{Ciprian Alexandru-Caragea}
\date{9/8/24}

\begin{document}
\frontmatter
\maketitle
\ifdefined\Shaded\renewenvironment{Shaded}{\begin{tcolorbox}[borderline west={3pt}{0pt}{shadecolor}, enhanced, interior hidden, boxrule=0pt, frame hidden, sharp corners, breakable]}{\end{tcolorbox}}\fi

\renewcommand*\contentsname{Cuprins}
{
\setcounter{tocdepth}{1}
\tableofcontents
}
\listoffigures
\listoftables
\mainmatter
\bookmarksetup{startatroot}

\hypertarget{despre-autor}{%
\chapter*{Despre autor}\label{despre-autor}}
\addcontentsline{toc}{chapter}{Despre autor}

\markboth{Despre autor}{Despre autor}

\setcounter{page}{3}

\textbf{Ciprian Alexandru-Caragea} este conferenţiar universitar la
Facultatea de Management Financiar, Universitatea Ecologică din
București și Analist de Date la diverse instituții internaționale.\\

\begin{wrapfigure}{r}{0.3\textwidth}
  \begin{center}
    \includegraphics[width=0.3\textwidth]{images/Ciprian_DGINS2018.jpg}
  \end{center}
\end{wrapfigure}

Titlul de doctor în Economie l-a obţinut sub egida Academiei Române,
Institutul de Economie Naţională.\\
A participat la un program de studii postdoctorale în care a implementat
utilizarea software-ului R ca instrument de analiză a evoluției
indicilor bursieri.\\
Activitatea sa didactică se concentrează, în principal, în domeniul
burselor de valori, prin cursuri și seminarii la programele de licență
și masterat (Piețe de capital, Managementul Portofoliului, Piețe
internaționale de capital).\\
A participat la diverse proiecte de cercetare, workshop-uri, conferințe
naționale și internaționale. Activitatea de cercetare a fost pusă în
valoare prin publicarea studiilor în reviste din țară și din Europa,
precum și în baze de date internaționale recunoscute (RePEC, DOAJ,
EBSCO).\\
În prezent, în cadrul Institului Național de Statistică, participă ca
expert în proiecte BigData și utilizează software-ul de analiză
statistică R pentru Data cleaning, Data Matching, Web Scraping, analize
de date și vizualizare, Data mining, Data integration, data processing,
data validation, dar și utilizarea datelor din sursele administrative
pentru realizarea de statistici oficiale.

\href{http://www.researcherid.com/rid/V-2168-2017}{ResearcherID:
V-2168-2017}\\
https://orcid.org/0000-0001-8215-6671

\bookmarksetup{startatroot}

\hypertarget{introducere}{%
\chapter*{Introducere}\label{introducere}}
\addcontentsline{toc}{chapter}{Introducere}

\markboth{Introducere}{Introducere}

\begin{quote}
``\ldots acum nu mai e nimic nou de descoperit;\\
tot ce rămâne e doar măsurătoarea din ce în ce mai precisă''

--- Lord Kelvin (1894)
\end{quote}

Cartea tipărită merită răsfoită. Trăim în vremea în care internetul
facilitează comunicarea globală, informația fiind disponibilă oricând și
oricum. Toată lumea, de la oameni de știință și până la copii de vârstă
școlară primesc și oferă informații și propagă idei pe calea
internetului. Tirajele publicațiilor, cărților și manualelor tipărite
sunt în scădere în întreaga lume, în timp ce postările online captează
atenția omenirii.\\
Obiectivul principal al cărții pe care o propun este de a fi un ghid
cuprinzător, în termeni de concepte și tehnici, reprezentativ și, mai
ales, practic, în ceea ce privește utilizarea instrumentelor software de
analiză statistică, R fiind principalul software utilizat pentru
aplicațiile propuse. Ca abordare generală, cartea prezintă principalele
concepte utilizate în statistică, cu exemple și explicații descriptive.
Exemplele din viața economică - cele mai multe dintre ele bazate pe date
statistice reale - problemele rezolvate, dar și cele propuse, acoperă o
arie cuprinzătoare de tematici, cititorul având șansa de a fi introdus
în sfera aplicativă a conceptelor teoretice parcurse.\\
Cartea este destinată tuturor celor care doresc să înțeleagă, prin
mijloace științifice, fenomenele economice și sociale, sub aspectul
măsurării cantitative și din perspectiva determinării cauzale. Deși se
adresează, în principal, studenților care se pregătesc să devină
specialiști în științele economice, lucrarea este utilă și celor care
își propun să cunoască un domeniu atât de frumos și de captivant. Tocmai
nevoia de informații, din ce în ce mai complexe, dar și posibilitățile
de calcul avansat cu ajutorul soft-urilor tot mai performante, au condus
la crearea unui bazin imens de date care pot fi cu ușurință exploatate
pe baza analizei statistice. Poate că acesta este și motivul pentru care
statistica rămâne o disciplină percepută ca fiind adesea prea
matematizată, destinată specialiștilor. Pentru mulți cititori, mai ales
dintre cei care nu au o formare bazată pe un aparat matematic, studiul
fenomenelor economice prin metode statistice și matematice, presupune un
efort deosebit. Din acest motiv, am încercat să tratez aspectele
teoretice, dar și problemele cu aplicație practică din sfera economică,
într-o manieră simplă, accesibilă. Așadar, lucrarea are menirea de a
facilita înțelegerea conceptelor fundamentale cu care operează
statistica, utilizarea adecvată a metodelor de analiză statistică,
precum și interpretarea corectă a rezultatelor, în vederea cunoașterii
modului de manifestare a fenomenelor.

\begin{flushright}
Nicoleta Caragea \\
Septembrie, 2018
\end{flushright}

\newpage

\hypertarget{quarto}{%
\section*{Quarto}\label{quarto}}
\addcontentsline{toc}{section}{Quarto}

\markright{Quarto}

Această carte a fost editată cu ajutorul pachetului R \textbf{bookdown}
(\protect\hyperlink{ref-xie2015}{\textbf{xie2015?}}).\\
Cartea are la bază manualul \emph{Statistică - concepte și metode de
analiză a
datelor}(\protect\hyperlink{ref-caragea2015}{\textbf{caragea2015?}}).

Pachetul R \textbf{bookdown} este integrat R Markdown
(http://rmarkdown.rstudio.com). Documentele elaborate pe baza acestui
tip de instrumentar de editare sunt pe deplin reproductibile și dau
posibilitatea creării unor formate de ieșire diverse
(PDF/HTML/Word/\ldots). Informații suplimentare referitoare la
utilizarea pachetului \textbf{bookdown} se pot găsi la adresa:
https://bookdown.org.

\includegraphics[width=0.2\textwidth]{images/logo.png}

\hypertarget{informaux21bii-despre-software}{%
\section*{Informații despre
software}\label{informaux21bii-despre-software}}
\addcontentsline{toc}{section}{Informații despre software}

\markright{Informații despre software}

Software-ul \textsf{R} a devenit în prezent unul dintre cele mai
utilizate instrumente de analiză statistică, fiind utilizat în
statisticile oficiale, în mediile universitare și de cercetare
academică, dar și în mediul de afaceri. Acest manual este destinat
tuturor celor care doresc să învețe statistica, fiind un material
introductiv de studiu, care prezintă un spectru larg de exemple,
prezentări grafice și analiză a datelor, dezvoltate cu ajutorul
\textsf{R}.\\
Aplicațiile din această carte utilizează \textsf{R}, ceea ce înseamnă că
pentru reproducerea acestora va fi nevoie de instalarea \textsf{R} pe
calculatorul pe care lucrați.\\
\textsf{R} este un sistem pentru analize statistice și reprezentare
grafică creat de către Ross Ihaka și Robert Gentleman, profesori de
statistică la Universitatea Auckland din Noua
Zeelandă\footnote{Ihaka R. \& Gentleman R. 1996. R: a language for data analysis and graphics. {\it Journal of Computational and Graphical Statistics} 5: 299--314.}.\\
\index{limbajul S}\textsf{R} este considerat un dialect al limbajului
\textsf{S} creat de AT\&T Bell Laboratories. \textsf{S} este disponibil
sub forma software-ului S-PLUS, comercializat de compania Insightful.
Există diferențe importante între cele două limbaje, \textsf{R} și
\textsf{S}: acestea sunt documentate de către Ihaka \& Gentleman (1996)
sau se regăsesc în
R-FAQ\footnote{\href{http://cran.r-project.org/doc/FAQ/R-FAQ.html\#What-are-the-differences-between-R-and-S_003f}{R-FAQ}}.\\
Astfel, numele limbajului R provine de la inițiala prenumelui
creatorilor, dar este totodată și un omagiu adus limbajului
\textsf{S}.\\
În primul rând, \textsf{R} este open-source, fiind distribuit în mod
gratuit sub licență
\textit{GNU - General Public Licence}\footnote{\href{http://www.gnu.org/}{GNU}};
dezvoltarea și distribuirea sunt în grija câtorva profesori și
statisticieni, afiliați companiilor și universităților, cunoscuți sub
denumirea generică de \textit{R Development Core Team}.\\
Conform filosofiei
\textit{GNU}\footnote{\href{http://www.gnu.org/philosophy/free-sw.ro.html\#exportcontrol}{GNU Philosophy}},
software-ul open-source este caracterizat de libertatea acordată
utilizatorilor săi de a-l utiliza, copia, distribui, studia, modifica și
îmbunătăți. Mai exact, este vorba de patru forme de libertate acordate
utilizatorilor(\protect\hyperlink{ref-dusa2015}{\textbf{dusa2015?}}):

\begin{itemize}
\item Libertatea de a utiliza programul, în orice scop (libertatea 0);
\item Libertatea de a studia modul de funcționare a programului, și de a-l adapta nevoilor proprii (libertatea 1). Accesul la codul-sursă este o precondiție pentru aceasta;
\item Libertatea de a redistribui copii, în scopul ajutorării aproapelui tău (libertatea 2);
\item Libertatea de a îmbunătăți programul, și de a pune îmbunătățirile la dispoziția publicului, în folosul întregii societăți (libertatea 3). Accesul la codul-sursă este o precondiție pentru aceasta.
\end{itemize}

Faptul că este gratuit atrage automat avantajul competitiv în fața altor
software-uri de analiză statistică, precum Stata, SAS și SPSS. Astfel,
costurile alocate licenței de software dispar. \textsf{R} este denumit
de către Norman Nie, unul dintre fondatorii SPSS și CEO al Revolution
Analytics, ``cel mai puternic și flexibil limbaj de programare
statistică din lume'' (în engleză
\textit{"the most powerful and flexible statistical programming language in the world"}).\footnote{\href{http://blog.revolutionanalytics.com/2010/10/r-is-hot.html}{Smith, D., 2010,"R is Hot", Revolution Analytics}}
Dovadă a succesului pe care \textsf{R} îl are în știința datelor, s-au
dezvoltat medii de integrare a acestuia în SAS și chiar SPSS. Este vorba
despre modulul
SAS/IML\footnote{\href{http://www.sas.com/en_us/software/analytics/iml.html\#close}{SAS/IML Module}},
care integrează limbajul \textsf{R} în SAS, și despre
\textit{translate2R}, un serviciu de translatare a codului SPSS direct
în \textsf{R} dezvoltat de compania
\textit{eoda}\footnote{\href{http://www.eoda.de/en/translate2R.html}{translate2R - eoda}}.
\textsf{R} are susținerea comunității științifice, dar și a multor
companii internaționale. Dintre acestea, menționăm: Google, Facebook,
Mozilla, Twitter, The New York Times, The Economist, NewScientist,
Lloyd's, Bing, Johnson\&Johnson, Pfizer, Shell, Bank of America,
Ford.\footnote{\href{http://www.revolutionanalytics.com/what-is-open-source-r/companies-using-r.php}{Revolution Analytics, "Companies Using R"}}
\index{open-source}\textsf{R} este susținut și de mediul academic.
Marile universități din lume sprijină \textsf{R}, la fel cum sprijină și
alte inițiative sau software-uri open-source, precum sistemul de operare
Linux sau sistemul de preparare a documentelor \LaTeX.

\bookmarksetup{startatroot}

\hypertarget{cap1}{%
\chapter{Concepte de bază în statistică}\label{cap1}}

\hypertarget{ce-este-statistica}{%
\section{Ce este statistica?}\label{ce-este-statistica}}

Termenul de ``statistică'' este unul familiar, pe care îl auzim sau
folosim în fiecare zi. Este asociat adesea cu calcule, cu un rezultat
exprimat procentual sau cu un tabel de date. Cu toate acestea, termenul
trebuie privit într-o accepțiune mai largă.\\
Poate v-ați întrebat deseori care este valoarea medie a cifrei de
afaceri sau care a fost creșterea medie a acesteia într-un interval de
timp, ce valoare a vânzărilor ar putea avea o firmă de telefonie mobilă
care ar lansa pe piață un nou produs sau serviciu, sau care ar fi cel
mai potrivit loc de muncă disponibil când veți absolvi facultatea. Cine
va fi ales noul președinte al țării la următoarele alegeri
prezidențiale? Sau poate vă întrebați care ar fi contribuția creșterii
nivelului de educație al populației ocupate la dezvoltarea economică a
țării.\\
Răspunsurile la astfel de întrebări, dar și la multe altele, vin din
înțelegerea corectă a valorilor numerice înregistrate de o variabilă
căreia îi vom spune generic ``de interes'', a fluctuațiilor sau a
tendinței de creștere sau de scădere a acestor valori într-o anumită
perioadă de timp.\\
Conform dicționarelor publicate de institutele de cercetare din România,
\textbf{statistica} este o \emph{știință care, folosind calculul
probabilităților, studiază fenomenele și proceselor de tip colectiv (din
societate, natură etc.) din punct de vedere cantitativ în scopul
descrierii acestora și a descoperirii legilor care guvernează
manifestarea lor}\footnote{Academia RPR \emph{Dicționar Enciclopedic
  Român}, București: Editura Politică, 1962-1966; Academia Română,
  Institutul de Lingvistică, Iorgu Iordan, \emph{Dicționarul explicativ
  al limbii române} (DEX), București: Editura Univers Enciclopedic, 1998}.
Altfel definită, statistica este \emph{știința care se ocupă cu
descrierea și analiza numerică a fenomenelor de masă, dezvăluind
particularitățile lor de volum, structură, dinamică, conexiune, precum
și regularitățile sau legile ce le guvernează}\footnote{\emph{Mică
  enciclopedie de statistică}, Editura Științifică și Enciclopedică,
  București, 1985}.\\
În practică, metodele statistice furnizează multor domenii ale științei
un puternic set de instrumente pentru analiza datelor și pentru
interpretarea semnificației rezultatelor cercetărilor statistice.\\
Sensul termenului statistică poate avea mai multe conotații. Una dintre
ele este aceea de \emph{date} sau \emph{colecție de date}, referitoare
la un anumit fenomen sau domeniu. Într-o altă accepțiune, statistica
înseamnă activitatea de \emph{producere a datelor statistice}, de
culegere, prelucrare și prezentare a informației statistice sub diverse
forme (diseminare), la dispoziția utilizatorilor. Mai mult, termenul de
statistică se utilizează și pentru a desemna \emph{metodele statistice
aplicate} în cunoașterea fenomenelor de masă, sau teoria statistică,
implicând expunerea sistematică a conceptelor și metodelor de cercetare
statistică.\\
Statistica are ca \emph{obiect} studierea aspectelor cantitativ-numerice
ale \emph{fenomenelor de masă}, a dimensiunii, dinamicii și structurii
acestora, a raporturilor de interdependență și a altor aspecte care pot
fi caracterizate numeric. Fenomenele de masă sunt fenomene care se
produc sub acțiunea comună și repetată a unui număr mare de factori, cu
caracter sistematic sau întâmplător. Fenomenele de masă sunt
caracterizate printr-o mare diversitate, formele individuale de
manifestare diferind de la o unitate la alta în funcție de modul în care
se combină acțiunea acestor factori. Esența acestor fenomene poate fi
pusă în evidență numai prin studierea unui număr mare de cazuri. De
exemplu, productivitatea muncii, la nivelul economiei naționale, poate
fi cunoscută numai prin analiza informațiilor privind rezultatele
activității tuturor unităților din economia națională, respectiv
cantitatea de forță de muncă utilizată (dintr-o ramură de activitate sau
dintr-o unitate teritorială) sau a unei părți semnificative a
acestora.\\
\emph{Metoda statistică} este constituită dintr-un ansamblu de operații,
tehnici, procedee și metode de cercetare statistică. Specificitatea
metodei statistice constă în faptul că se bazează pe observarea unui
număr mare de cazuri și utilizează instrumente derivate din teoria
probabilităților, dezvoltate în cadrul statisticii matematice.

\hypertarget{uxeenceputul-demersului-statistic}{%
\section{Începutul demersului
statistic}\label{uxeenceputul-demersului-statistic}}

Înainte de a începe studiul statisticii este necesară o privire de
ansamblu asupra termenilor cu care operează statistica. Este cunoscut
faptul că, pentru organizarea unei cercetări corecte și cuprinzătoare,
este necesar să se folosească un limbaj științific specific disciplinei.
Acest lucru este valabil și pentru statistică, care a reușit să-și
elaboreze propriile noțiuni, concepte de bază pe care să le folosească
pe parcursul întregului demers statistic, de la stabilirea obiectivelor
cercetării și până la analiza și interpretarea rezultatelor. În acest
sens, cercetarea statistică operează cu câteva concepte specifice:
colectivitatea sau populația statistică, unitatea statistică,
caracteristicile sau variabilele statistice, datele statistice și
indicatorii statistici.\\
\emph{Colectivitatea} sau \emph{populația statistică} reprezintă
totalitatea elementelor sau a manifestărilor de aceeași natură, asupra
cărora se efectuează cercetarea. În funcție de obiectivul și
particularitățile cercetării, populația statistică poate fi formată din
persoane sau grupuri de persoane, din instituții, din obiecte, din
tranzacții economice, din evenimente etc.\\
De exemplu:

\begin{itemize}
\tightlist
\item
  O cercetare privind veniturile populației se realizează pe o
  colectivitate formată din gospodării;
\item
  O cercetare referitoare la nivelul de educație a populației -- se
  realizează pe o colectivitate formată din persoane;
\item
  Colectivitatea care face obiectul unei investigații privind producția
  și eficiența economică este alcătuită din unități economice;
\item
  Colectivitatea cercetată în cadrul unui studiu privind activitatea
  sistemului de sănătate cuprinde unități sanitare (toate spitalele,
  policlinicile, centrele de tratament etc.);
\item
  Studiul statistic al calității producției se realizează pe
  colectivități formate din obiecte (produse finite, semifabricate
  etc.);
\item
  Studiul statistic al natalității sau mortalității se realizează pe
  colectivități compuse din evenimente (nașteri sau decese).
\end{itemize}

Colectivitatea supusă cercetării poate fi totală (cercetare exhaustivă),
atunci când se fac înregistrări referitoare la toate elementele care
formează obiectul studiului, sau parțială (eșantion), atunci când se fac
înregistrări referitoare numai la o parte din această colectivitate.\\
Unitățile statistice reprezintă mulțimea numărabilă de elemente care
compun colectivitatea statistică. De exemplu, unitatea statistică a unei
cercetări referitoare la veniturile populației poate fi gospodăria sau
persoana. Unitatea statistică a unei cercetări privind calitatea
producției este produsul finit, semifabricatul sau piesa căreia i se
testează caracteristicile.\\
\emph{Unitățile statistice} pot fi simple sau complexe. Acestea din urmă
sunt formate din mai multe unități simple. O astfel de unitate este
gospodăria\footnote{Prin \textbf{gospodărie} se înțelege grupul de două
  sau mai multe persoane care locuiesc împreună în mod obișnuit, având,
  în general, legături de rudenie și care se gospodăresc (fac menajul)
  în comun, participând în totalitate sau parțial la formarea
  veniturilor și la cheltuirea lor. Persoana care nu aparține de o
  gospodărie și care declară că locuiește și se gospodărește singură se
  consideră gospodărie formată dintr-o singură persoană. Se consideră
  membri ai gospodăriei și persoanele plecate din localitate pentru o
  perioadă mai mare de 6 luni, care se află în țară sau străinătate,
  dacă acestea păstrează legături familiale cu gospodăria}. Unitățile pe
care se realizează cercetarea salariilor sunt, de asemenea unități
complexe: unitățile economice la care se face înregistrarea datelor.
Fiecare element al colectivității este purtătorul unei
\emph{caracteristici} supuse observării statistice. Caracteristica
statistică reprezintă acea proprietate/însușire care este comună tuturor
unităților unei colectivități statistice cercetate. Formele sau
nivelurile concrete ale acestora, denumite \emph{variante} sau
\emph{valori}, diferă de la o unitate la alta (sau în timp, în cazul
aceleiași unități) sub influența unui complex de factori. Numărul
unităților la care se înregistrează aceeași variantă sau valoare poartă
denumirea de \emph{frecvență} a variantei/valorii respective.\\
Caracteristicele statistice poartă numele și de variabile. Acestea pot
fi calitative sau cantitative.\\
Iată câteva exemple de variabile calitative în tabelele de mai jos:

\begin{table}[!h]
\centering
\caption{Exemple de variabile calitative}
\label{my-label}
\begin{tabular}{@{}ll@{}}
\toprule
\textit{Variabilă}   & \textit{Categorie}                                                                                                            \\ \midrule
Gen                  & \begin{tabular}[c]{@{}l@{}}masculin,\\ feminin\end{tabular}                                                                   \\
Mediu de rezidență   & \begin{tabular}[c]{@{}l@{}}urban,\\ rural\end{tabular}                                                                        \\
Formă juridică       & \begin{tabular}[c]{@{}l@{}}societate comercială pe acțiuni,\\ societate cu răspundere limitată,\\ regie autonomă\end{tabular} \\
Activitate economică & \begin{tabular}[c]{@{}l@{}}industrie,\\ agricultură,\\ comerț\end{tabular}                                                    \\
Statut ocupațional   & \begin{tabular}[c]{@{}l@{}}salariat,\\ agricultor,\\ șomer,\\ pensionar\end{tabular}                                          \\
Nivel de educație    & \begin{tabular}[c]{@{}l@{}}scăzut,\\ mediu,\\ superior\end{tabular}                                                           \\ \bottomrule
\end{tabular}
\end{table}

\textbf{R Code}

\begin{Shaded}
\begin{Highlighting}[]
\NormalTok{date\_educatie }\OtherTok{\textless{}{-}} \FunctionTok{read.csv}\NormalTok{(}\StringTok{"date/date\_educatie.csv"}\NormalTok{)}
\FunctionTok{head}\NormalTok{(date\_educatie)}
\end{Highlighting}
\end{Shaded}

\begin{verbatim}
  id varsta      gen mediu_rezidenta nivel_educatie
1  1     63 masculin           urban          mediu
2  2     55 masculin           rural         scazut
3  3     31 masculin           rural         scazut
4  4     56 masculin           urban          mediu
5  5     31  feminin           urban          mediu
6  6     63  feminin           rural       superior
\end{verbatim}

\begin{Shaded}
\begin{Highlighting}[]
\FunctionTok{summary}\NormalTok{(date\_educatie[}\DecValTok{2}\SpecialCharTok{:}\DecValTok{4}\NormalTok{])}
\end{Highlighting}
\end{Shaded}

\begin{verbatim}
     varsta          gen            mediu_rezidenta   
 Min.   :16.00   Length:70          Length:70         
 1st Qu.:33.00   Class :character   Class :character  
 Median :44.50   Mode  :character   Mode  :character  
 Mean   :43.79                                        
 3rd Qu.:55.00                                        
 Max.   :65.00                                        
\end{verbatim}

\textbf{Python Code}

\begin{Shaded}
\begin{Highlighting}[]
\ImportTok{import}\NormalTok{ pandas }\ImportTok{as}\NormalTok{ pd}
\NormalTok{date\_educatie }\OperatorTok{=}\NormalTok{ pd.read\_csv(}\StringTok{"date/date\_educatie.csv"}\NormalTok{)}
\BuiltInTok{print}\NormalTok{(date\_educatie.head())}
\end{Highlighting}
\end{Shaded}

\begin{verbatim}
   id  varsta       gen mediu_rezidenta nivel_educatie
0   1      63  masculin           urban          mediu
1   2      55  masculin           rural         scazut
2   3      31  masculin           rural         scazut
3   4      56  masculin           urban          mediu
4   5      31   feminin           urban          mediu
\end{verbatim}

\begin{Shaded}
\begin{Highlighting}[]
\BuiltInTok{print}\NormalTok{(date\_educatie.iloc[:, }\DecValTok{1}\NormalTok{].describe())}
\end{Highlighting}
\end{Shaded}

\begin{verbatim}
count    70.000000
mean     43.785714
std      14.613764
min      16.000000
25%      33.000000
50%      44.500000
75%      55.000000
max      65.000000
Name: varsta, dtype: float64
\end{verbatim}

\begin{Shaded}
\begin{Highlighting}[]
\BuiltInTok{print}\NormalTok{(date\_educatie.iloc[:, }\DecValTok{2}\NormalTok{:}\DecValTok{5}\NormalTok{].describe())}
\end{Highlighting}
\end{Shaded}

\begin{verbatim}
            gen mediu_rezidenta nivel_educatie
count        70              70             70
unique        2               2              3
top     feminin           rural         scazut
freq         39              41             25
\end{verbatim}

Variabilele cantitative sunt măsurabile și se exprimă numeric. Acestea
pot fi discrete sau continue.\\
De exemplu, nota la statistică, este o variabilă cantitativă, discretă.

\begin{table}[!h]
\centering
\caption{Exemple de variabile cantitative discrete}
\label{my-label}
\begin{tabular}{@{}lc@{}}
\toprule
\multicolumn{1}{c}{\textit{Variabilă}} & \multicolumn{1}{c}{\textit{Valoare}} \\ \midrule
\multirow{6}{*}{Nota la statistică}    & 10                                   \\
                                       & 9                                    \\
                                       & 8                                    \\
                                       & 7                                    \\
                                       & 6                                    \\
                                       & 5                                    \\ \cmidrule(l){2-2} 
\end{tabular}
\end{table}

Un exemplu de variabilă cantitativă continuă este venitul net al
salariaților unei firme.\\
Considerăm că se cunosc veniturile tuturor celor n=200 salariați.

\begin{table}[!h]
\centering
\caption{Exemple de variabile cantitative continue}
\label{my-label}
\begin{tabular}{@{}lc@{}}
\toprule
\multicolumn{1}{c}{\textit{Variabilă}} & \multicolumn{1}{c}{\textit{Valoare}} \\ \midrule
\multirow{6}{*}{Venit net}             & 3500                                  \\
                                       & 1800                                  \\
                                       & ...                                  \\
                                       & 6300                                  \\
                                       & 5450                                  \\ \cmidrule(l){2-2} 
\end{tabular}
\end{table}

Statistica operează, în principal, cu frecvențele diferitelor variante
ale caracteristicilor calitative, iar analiza statistică a acestor
caracteristici se realizează prin metode ce diferă de cele utilizate în
studiul variabilelor cantitative.\\
În cazul unei cercetări referitoare la activitatea economică, de
exemplu, se vor înregistra informații privind volumul producției,
cheltuielile de producție și profitul, numărul de salariați și dotarea
cu active corporale etc. din fiecare unitate economică.\\
Evaluarea situației existente pe piața forței de muncă, evoluția
fenomenelor de ocupare, șomaj și inactivitate va înregistra informații
referitoare la populația în vârstă de muncă, populația activă, populația
ocupată, numărul șomerilor etc. Un alt exemplu este cel al unei
cercetări referitoare la venituri, care va cuprinde, în principal,
informații numerice despre nivelul veniturilor de care beneficiază
gospodăriile (din activități salariale, din agricultură, din activități
pe cont propriu, din prestații sociale etc.), despre impozitele și
contribuțiile de asigurări sociale plătite de gospodării, despre mărimea
gospodăriei, vârsta membrilor acesteia etc.\\
Prin natura lor, variabilele cantitative se grupează în:

\begin{itemize}
\tightlist
\item
  variabile discrete - pot lua numai anumite valori (exprimate prin
  numere întregi);
\item
  variabile continue - pot lua orice valori într-un interval finit sau
  infinit.
\end{itemize}

Numărul salariaților unei firme sau numărul de persoane care compun o
gospodărie sunt variabile discrete, în timp ce valoarea producției,
nivelul salariilor sau nivelul veniturilor sunt variabile continue
(chiar dacă nivelurile înregistrate ale acestora apar ca valori ale unor
variabile discrete). În practică, variabilele discrete sunt tratate ca
fiind continue, dacă valorile lor posibile sunt numeroase (de ex.,
numărul salariaților unităților economice).\\
\emph{Datele statistice} exprimă valori ale unor caracteristici
cantitative ale unităților statistice, precum și ale unor grupe ale
colectivității sau ale colectivității în ansamblul ei. Informațiile
numerice sunt date statistice numai dacă sunt definite sub aspectul
conținutului, al unității de măsură, al identificării
unității/grupului/colectivității la care se referă, precum și al
localizării în spațiu și în timp. Datele se obțin fie direct prin
observare, fie sunt rezultate ale prelucrării datelor primare. De
exemplu, date statistice sunt nivelurile cifrei de afaceri înregistrate
de firmele X și Y din București, în luna septembrie 2014.\\
\emph{Indicatorii statistici} reprezintă expresia numerică ce exprimă
manifestările unor fenomene economice sau sociale, caracterizând nivelul
și structura, modificările în timp și variația spațială, precum și
interdependențele dintre ele.\\
Indicatorii statistici sunt rezultate ale prelucrării informațiilor
primare înregistrate în cursul unei observări statistice sau a unor
informații cuprinse în publicații statistice sau baze de date. De
exemplu, modificarea cifrei de afaceri a firmei X în luna septembrie
2014 față de aceeași lună a anului precedent este un indicator
statistic.\\
Pentru cunoașterea fenomenelor de masă, indicatorii statistici
îndeplinesc mai multe funcții și anume: de măsurare, de comparare, de
analiză, de sinteză; de estimare, de verificare a ipotezelor și/sau de
testare a semnificației parametrilor utilizați.\\
Simpla enumerare a principalelor funcții\footnote{Pentru elaborarea şi
  utilizarea corectă a indicatorilor statistici este esenţială
  îndeplinirea unor cerinţe de principiu, generale. În acest sens, Yule
  (1945) precizează condiţiile care ar trebui să le îndeplinească un
  astfel de indicator şi anume:\\
  - să fie definit în mod obiectiv, independent de dorinţa
  utilizatorului;\\
  - să depindă determinarea sa de toate valorile individuale
  înregistrate;\\
  - să aibă o semnificaţie concretă, uşor de înţeles chiar şi de
  nespecialişti;\\
  - să fie simplu şi rapid de calculat;\\
  - să fie puţin sensibil la fluctuaţiile de selecţie (să nu prezinte
  valori puternic diferite, dacă se calculează pe baza mai multor
  eşantioane, de acelaşi volum, extrase prin acelaşi procedeu din
  aceeaşi colectivitate);\\
  - să se preteze la calcule algebrice (să poată fi utilizat în operaţii
  de comparare a mai multor serii statistice sau în operaţii de
  agregare/dezagregare).} ale indicatorilor statistici pune în evidență
o multitudine de aspecte care trebuie avute în vedere la elaborarea și
folosirea acestora în analiză, inclusiv stabilirea condițiilor și
limitelor în care pot fi utilizați indicatorii statistici în raport cu
conținutul specific al fenomenelor, al surselor de informație de care se
dispune, cu scopul cercetării.\\
Indicatorii statistici se pot grupa în indicatori primari și derivați:\\
1. Indicatori primari (mărimi absolute) -- exprimă direct, general
nivelul caracteristicii cercetate. Se pot obține prin înregistrarea
directă, centralizarea datelor sau prin însumarea parțială sau totală a
datelor individuale; prezintă o capacitate relativ limitată de descriere
a fenomenului/procesului analizat, și nu permite realizarea unor
aprecieri calitative, însă reprezintă punctul de plecare al analizei
statistice;\\
2. Indicatori derivați -- se obțin prin prelucrarea indicatorilor
primari (absoluți) și fac posibilă analiza aspectelor calitative ale
fenomenelor și proceselor analizate (de exemplu: mărimi relative, mărimi
medii, indicatori ai variației, indici, indicatori ai corelației, etc.).

\hypertarget{mux103rimi-relative}{%
\section{Mărimi relative}\label{mux103rimi-relative}}

Mărimile relative constituie una dintre principalele categorii de
indicatori statistici utilizați în analiza economică. Indicatorii
statistici reprezintă expresii numerice ale unor fenomene, procese sau
categorii economico-sociale, definite în timp, în spațiu și
instituțional-organizatoric.\\
Statistica operează cu indicatori primari și derivați.\\
\textbf{Indicatorii primari} sunt mărimi absolute care exprimă nivelul
caracteristicii cercetate, obținându-se prin înregistrare directă sau
prin însumarea datelor individuale. Indicatorii primari se exprimă în
unități de măsură, naturale sau monetare, specifice caracteristicii
observate: bucăți, tone, kilograme, metri, metri pătrați, dolari, Euro,
lei, mii lei, milioane lei etc. Un astfel de indicator este, de exemplu:
cifra de afaceri (în milioane lei) înregistrată de o firmă având ca
principală activitate comercializarea carburanților și cea rezultată
prin însumare la nivelul întregii țări; volumul producției la nivelul
unei firme sau la nivel național, numărul populației unei localități, a
unui județ și a țării; numărul salariaților unei unități economice
(societate comercială sau instituție publică), al celor dintr-o anumită
ramură de activitate sau dintr-o unitate teritorială (localitate sau
județ) și al tuturor salariaților din economia națională; veniturile și
cheltuielile bugetelor de asigurări sociale (în miliarde lei); volumul
exportului sau importului realizat la o anumită grupă de produse, cu o
anumită țară sau cu toate țările (în milioane dolari sau Euro).\\
\textbf{Indicatorii derivați} se obțin în urma prelucrării datelor
primare, prin aplicarea diferitelor metode și procedee de calcul
statistic. Există o mare diversitate de indicatorii derivați, cu metode
variate de calcul, de la operații simple de împărțire sau scădere la
calcule bazate pe formule complexe. În categoria indicatorilor derivați
intră mediile, quantilele, indicatorii variației și ai inegalității,
indicatorii corelației, indicii etc. O parte a acestora au caracter
absolut și se exprimă în unități de măsură naturale sau monetare
specifice caracteristicilor care au stat la baza calculului lor. Altele
de exprimă sub formă de coeficienți sau procente. Mărimile relative
formează una dintre categoriile de indicatori derivați cele mai frecvent
utilizate în analiză.\\
\textbf{Mărimile relative} rezultă din compararea a doi indicatori, prin
raportarea unuia, numit termen de comparat, la altul numit bază de
comparație. Rezultatul obținut este un număr întreg sau fracționar, care
arată câte unități din indicatorul de comparat revin la o unitate din
indicatorul bază de raportare. În cazul celor mai multe mărimi relative,
rezultatul raportului capătă relevanță în analiză prin multiplicarea cu
10k (respectiv cu 10, 100, 1000, 10000, 100000, \ldots.., corespunzător
lui k = 1, 2, 3, 4, 5, \ldots). Mărimile relative calculate ca raport a
doi indicatori absoluți similari, exprimați în aceeași unitate de
măsură, iau forma unor coeficienți rezultați direct din împărțire sau se
exprimă sub formă de procente (\%, dacă rezultatul raportului este
înmulțit cu 100), promile (‰, dacă rezultatul este multiplicat cu 1000),
prodecimile (în cazul înmulțirii cu 10000) sau procentimile (în cazul
înmulțirii cu 100000). Mărimile relative calculate ca raport a doi
indicatori diferiți, exprimați în unități de măsură diferite, sunt
exprimate în unități de măsură compuse din unitățile de măsură ale celor
doi indicatori. Dacă asigurarea relevanței rezultatului impune
multiplicarea raportului cu 100, 1000 etc., aceasta este reflectată în
unitatea de măsură a indicatorului rezultat. Principalele tipuri de
mărimi relative sunt: mărimile relative de structură, mărimile relative
de coordonare, mărimile relative de intensitate, mărimile relative ale
dinamicii, mărimile relative ale planului.

\hypertarget{mux103rimi-relative-de-structurux103}{%
\subsection{Mărimi relative de
structură}\label{mux103rimi-relative-de-structurux103}}

Mărimile relative de structură reprezintă raportul dintre un element sau
un grup de elemente ale unei colectivități și întreaga colectivitate,
calculat pe baza unei caracteristici de volum sau a frecvenței. Mărimea
relativă de structură este raportul dintre o parte și întregul de care
aparține această parte. Mărimile relative de structură sunt exprimate,
în principal, sub formă procentuală, indicând câte unități revin părții
analizate la fiecare 100 de unități ale întregului. Dacă sunt exprimate
sub formă de coeficienți, acestea arată ce fracțiune din întreg revine
părții analizate (0,25, adică 25 de sutimi un sfert; 0,10 sau o zecime;
0,33(3) sau o treime etc).\\
Dacă se calculează pe baza unei caracteristici de volum (volumul
producției, al cheltuielilor de producție, al profitului, al importului
sau exportului, al veniturilor etc), mărimea relativă de structură
poartă denumirea de pondere sau greutate specifică. Aceasta exprimă
raportul în care se află volumul aferent părții analizate (xi) față de
volumul însumat al tuturor părților care compun întregul:

\[g = \frac{x_{i}}{\sum_{i=1}^n x_i}100\] Observații:\\
* Suma ponderilor tuturor componentelor este egală cu 100\%.\\
** Diferența dintre doi indicatori calculați sub formă procentuală este
exprimată în puncte procentuale (p.p.).

Exemple: 1. Producția industrială realizată în anul t, pe principalele
grupe de activități, a înregistrat următoarele valori, conform tabelului
de mai jos:

\begin{table}[!h]
\centering
\caption{Producția industrială în anul t}
\label{my-label}
\begin{tabular}{@{}lr@{}}
\toprule
                                                                                             & \multicolumn{1}{c}{\textit{Q (mil.lei)}} \\ \midrule
\textbf{Industria extractivă}                                                                & \textit{55403,4}                         \\
\textbf{Industria prelucrătoare}                                                             & \textit{769938,6}                        \\
\textbf{\begin{tabular}[c]{@{}l@{}}Energie electrică și\\ termică, gaze și apă\end{tabular}} & \textit{141103,0}                        \\
\textit{Total}                                                                               & \textit{966445,0}                        \\ \bottomrule
\end{tabular}
\end{table}  

În anul t, ponderea industriei extractive în volumul total al producției
industriale a fost de:

\[g = \frac{x_{i}}{\sum_{i=1}^n x_i}100 =\frac{55403,4}{966445}100 = 5,7\%\]

Ponderile celorlalte două grupe de activități se calculează în același
mod, astfel încât structura producției industriale are următoarea
configurație:

\begin{table}[!h]
\centering
\caption{Structura producției industriale în anul t}
\label{my-label}
\begin{tabular}{@{}lc@{}}
\toprule
                                                                                             & \multicolumn{1}{c}{\textit{g (\%)}} \\ \midrule
\textbf{Industria extractivă}                                                                & \textit{5,7}                         \\
\textbf{Industria prelucrătoare}                                                             & \textit{79,7}                        \\
\textbf{\begin{tabular}[c]{@{}l@{}}Energie electrică și\\ termică, gaze și apă\end{tabular}} & \textit{14,6}                        \\
\textit{Total}                                                                               & \textit{100,0}                        \\ \bottomrule
\end{tabular}
\end{table}  

Mărimile relative de structură calculate pentru seriile de distribuție
de frecvențe, numite \textbf{frecvențe relative} sau \textbf{ponderi},
exprimă raportul în care se află numărul unităților din fiecare grupă
(ni) față de numărul unităților din întreaga colectivitate.

\begin{enumerate}
\def\labelenumi{\arabic{enumi}.}
\setcounter{enumi}{1}
\tightlist
\item
  În anul t, s-a înregistrat următoarea distribuție a salariaților după
  nivelul salariilor:
\end{enumerate}

\begin{table}[!h]
\centering
\caption{Distribuția salariaților după nivelul salariului brut, în anul t}
\label{my-label}
\begin{tabular}{@{}rcc@{}}
\toprule
\multicolumn{1}{l}{} & \textbf{\begin{tabular}[c]{@{}c@{}}Grupe de salarii brute\\ realizate (lei)\end{tabular}} & \textbf{\begin{tabular}[c]{@{}c@{}}Pondere\\ în numărul total de salariați n’ (\%)\end{tabular}} \\ \midrule
1                    & Până la 350                                                                               & \textit{23,6}                                                                                    \\
2                    & 350-400                                                                                   & \textit{8,1}                                                                                     \\
3                    & 400-500                                                                                   & \textit{12,6}                                                                                    \\
4                    & 500-700                                                                                   & \textit{17,9}                                                                                    \\
5                    & 700-1000                                                                                  & \textit{16,1}                                                                                    \\
6                    & 1000-1500                                                                                 & \textit{12,5}                                                                                    \\
7                    & 1500-2000                                                                                 & \textit{4,6}                                                                                     \\
8                    & 2000-3000                                                                                 & \textit{2,7}                                                                                     \\
9                    & 3000-4000                                                                                 & \textit{1,0}                                                                                     \\
10                   & 4000-5000                                                                                 & \textit{0,4}                                                                                     \\
11                   & Peste 5000                                                                                & \textit{0,5}                                                                                     \\
\multicolumn{1}{l}{} & \textit{\textbf{Total}}                                                                   & \textit{\textbf{100,0}}                                                                          \\ \bottomrule
\end{tabular}
\end{table}    

Dacă numărul total al salariaților a fost de 3814 mii în anul t, poate
fi estimat și numărul celor cuprinși în prima grupă (în mii):

\[n_{1} = \frac{n^{'}_{1}}{100}N =\frac{23,6}{100}3814 = 900\]

În mod similar se determină și numărul salariaților cuprinși în
celelalte grupe. De exemplu, pentru ultima grupă, a unsprezecea:

\[n_{11} = \frac{n^{'}_{11}}{100}N =\frac{0,5}{100}3814 = 19,07\]

Așadar, numărul salariaților cu un salariu mai mare de 5000 este de 19
mii.

\hypertarget{mux103rimi-relative-de-intensitate}{%
\subsection{Mărimi relative de
intensitate}\label{mux103rimi-relative-de-intensitate}}

Mărimile relative de coordonare se utilizează pentru a compara două
niveluri ale aceluiași indicator, niveluri care privesc două grupe/părți
(A și B) ale aceleiași populații statistice sau două populații (A și B)
de același tip situate în spații diferite. Calculul mărimilor relative
de coordonare se poate face raportând nivelul înregistrat în cazul
populației/grupei A la cel înregistrat în cazul populației/grupei B sau,
invers, raportând nivelul aferent populației B la cel corespunzător
populației/grupei A:

\[k_{{A}/{B}} = \frac{x_{A}}{x_{B}}\]

\[k_{{B}/{A}} = \frac{x_{B}}{x_{A}}\]

Rezultatul este supraunitar într-una dintre variantele de calcul și
subunitar în cealaltă. Dacă xA \textgreater{} xB, raportul kA/B este
supraunitar și arată de câte ori este mai mare nivelul corespunzător
populației A, comparativ cu B. În acest caz, raportul invers kB/A este
subunitar, indicând ce fracțiune reprezintă nivelul înregistrat de
populația B în raport cu nivelul A. Indicatorul subunitar este mai
relevant dacă este exprimat procentual.

Exemple:\\
1. În anul t, raportul dintre produsul intern brut pe locuitor al
Italiei și cel al României (calculate în dolari la paritatea puterii de
cumpărare) a fost de 3,3.

\[k_{{Italia}/{România}} = \frac{PIB_{Italia}}{PIB_{România}}\]
\[k_{{Italia}/{România}} = \frac{23300}{7100} = 3,3\]

Rezultatul indică un nivel de bunăstare de 3,3 ori mai înalt în Italia
decât în România.

Raportul invers:

\[k_{{România}/{Italia}} = \frac{PIB_{România}}{PIB_{Italia}} = 0,304\]
arată că nivelul de bunăstare înregistrat în România reprezintă numai
30,4\% în raport cu cel atins de Italia.

\hypertarget{mux103rimi-relative-de-coordonare}{%
\subsection{Mărimi relative de
coordonare}\label{mux103rimi-relative-de-coordonare}}

Mărimile relative de intensitate se determină prin raportarea a doi
indicatori absoluți (x și y) de natură diferită, dar care se află într-o
relație de interdependență. În calculul indicatorului relativ,
rezultatul raportului este multiplicat cu 10k, dacă această operație
este necesară pentru asigurarea relevanței:

\[r = \frac{x}{y}10^{k}\]

Indicatorii utilizați în calculul mărimilor relative de intensitate sunt
exprimați, în general, în unități de măsură diferite, iar rezultatul
este exprimat într-o unitate de măsură compusă. Din categoria mărimilor
relative de intensitate fac parte următorii indicatori:

\begin{enumerate}
\def\labelenumi{\alph{enumi})}
\tightlist
\item
  \textbf{productivitatea muncii} (w), calculată ca raport între volumul
  fizic sau valoarea producției (Q) și consumul de muncă (T, dacă este
  exprimat în unități de timp de muncă; N, dacă este exprimat în număr
  de persoane ocupate/salariați):
\end{enumerate}

\[w = \frac{Q}{N}\] \[w = \frac{Q}{T}\] sau ca raport între consumul de
muncă (T) și volumul producției (Q):

\[w = \frac{T}{Q}\]

\begin{enumerate}
\def\labelenumi{\alph{enumi})}
\setcounter{enumi}{1}
\tightlist
\item
  \textbf{consumul specific} (c), calculat ca raport între consumul
  total de materii prime, materiale și energie (C) și volumul producției
  (Q):
\end{enumerate}

\[c = \frac{C}{Q}\]

\begin{enumerate}
\def\labelenumi{\alph{enumi})}
\setcounter{enumi}{2}
\tightlist
\item
  \textbf{randamentele în agricultură}, calculate ca raport între
  producția vegetală realizată și suprafețele cultivate cu diferite
  culturi (kilograme de grâu sau de porumb boabe pe hectar, de exemplu)
  sau între producția animală și numărul animalelor;\\
\item
  \textbf{indicatorii demografici}: ratele natalității și mortalității,
  nupțialității și divorțialității, calculate ca raport (multiplicat cu
  1000) între numărul copiilor născuți vii, al deceselor, al
  căsătoriilor și divorțurilor înregistrate în cursul unui an și numărul
  populației; rata mortalității infantile și juvenile, calculată ca
  raport (multiplicat cu 1000) între numărul copiilor sub un an,
  respectiv sub 5 ani, decedați și numărul copiilor născuți vii în
  cursul unui an etc.\\
  O altă categorie de mărimi relative de intensitate este formată din
  cele calculate pe baza a doi indicatori diferiți sub aspectul
  conținutului economic, dar exprimați în aceeași unitate de măsură.
  Indicatorul relativ rezultat din calcul are forma unei rate
  procentuale:\\
\end{enumerate}

\begin{itemize}
\tightlist
\item
  rata de activitate și rata de ocupare, calculate ca raport între
  populația activă, respectiv ocupată și populația în vârstă de muncă;\\
\item
  rata șomajului, calculată ca raport între numărul șomerilor și numărul
  populației active;\\
\item
  rata sărăciei, calculată ca raport între populația săracă și populația
  totală;\\
\item
  rata rentabilității;\\
\item
  rata dobânzii etc.
\end{itemize}

\hypertarget{mux103rimi-relative-ale-dinamicii-ux219i-ale-planului}{%
\subsection{Mărimi relative ale dinamicii și ale
planului}\label{mux103rimi-relative-ale-dinamicii-ux219i-ale-planului}}

Mărimile relative ale dinamicii se utilizează pentru caracterizarea
evoluției în timp a fenomenelor social-economice. Cele mai importante
componente ale acestei categorii de mărimi relative sunt indicii și
ritmurile. Indicii dinamicii se determină ca raport între nivelurile
înregistrate de o variabilă într-o anumită perioadă sau la un moment dat
(\(x_t\)) și într-o perioadă sau la un moment anterior, considerat bază
de comparație (\(x_0\)):

\[i_{t/0} = \frac{x_{t}}{x_{0}}\]

Rezultatul raportului arată de câte ori a crescut nivelul variabilei în
perioada t față de perioada de bază (dacă sunt supraunitari) sau la ce
fracțiune din nivelul de bază a scăzut nivelul variabilei (dacă sunt
subunitari). Forma de prezentare a indicilor este cea procentuală.\\
Ritmul reprezintă raportul dintre sporul absolut al variabilei în
perioada \texttt{t} față de perioada de bază și nivelul înregistrat în
perioada de bază:

\[r_{t/0} = \frac{x_{t}-x_{0}}{x_{0}}100\]

și arată cu cât a crescut sau a scăzut nivelul variabilei în raport cu
nivelul de bază. Mărimile relative ale planului sunt utilizate pentru
analiza relației dintre nivelul înregistrat de un indicator în perioada
de bază (x0), cel programat (\(x_{pl}\)) și cel realizat în perioada
curentă (x1). Se utilizează doi indici ai planului, ambii calculați ca
raport între nivelurile indicatorului analizat:\\
• indicele sarcinii de plan, determinat prin raportarea nivelului
planificat la cel realizat în perioada de bază:

\[i_{plan/0} = \frac{x_{plan}}{x_{0}}100\] • indicele realizării
planului, determinat prin raportarea nivelului realizat la cel
planificat pentru perioada curentă:

\[i_{1/plan} = \frac{x_{1}}{x_{plan}}100\]

Cei doi indici arată proporția modificării programate a fenomenului
analizat, respectiv proporția în care a fost realizat nivelul programat.
Ambii indici ai planului se calculează sub formă procentuală.

\bookmarksetup{startatroot}

\hypertarget{summary}{%
\chapter*{Summary}\label{summary}}
\addcontentsline{toc}{chapter}{Summary}

\markboth{Summary}{Summary}

In summary, this book has no content whatsoever.

\begin{Shaded}
\begin{Highlighting}[]
\DecValTok{1} \SpecialCharTok{+} \DecValTok{1}
\end{Highlighting}
\end{Shaded}

\begin{verbatim}
[1] 2
\end{verbatim}

\bookmarksetup{startatroot}

\hypertarget{bibliografie}{%
\chapter*{Bibliografie}\label{bibliografie}}
\addcontentsline{toc}{chapter}{Bibliografie}

\markboth{Bibliografie}{Bibliografie}

\hypertarget{refs}{}
\begin{CSLReferences}{0}{0}
\end{CSLReferences}


\backmatter

\end{document}
