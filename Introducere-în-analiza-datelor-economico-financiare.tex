% Options for packages loaded elsewhere
\PassOptionsToPackage{unicode}{hyperref}
\PassOptionsToPackage{hyphens}{url}
%
\documentclass[
  11pt,
  b5paper,
  nottoc]{book}

\usepackage{amsmath,amssymb}
\usepackage{lmodern}
\usepackage{iftex}
\ifPDFTeX
  \usepackage[T1]{fontenc}
  \usepackage[utf8]{inputenc}
  \usepackage{textcomp} % provide euro and other symbols
\else % if luatex or xetex
  \usepackage{unicode-math}
  \defaultfontfeatures{Scale=MatchLowercase}
  \defaultfontfeatures[\rmfamily]{Ligatures=TeX,Scale=1}
  \setmonofont[Scale=0.7]{Source Code Pro}
\fi
% Use upquote if available, for straight quotes in verbatim environments
\IfFileExists{upquote.sty}{\usepackage{upquote}}{}
\IfFileExists{microtype.sty}{% use microtype if available
  \usepackage[]{microtype}
  \UseMicrotypeSet[protrusion]{basicmath} % disable protrusion for tt fonts
}{}
\makeatletter
\@ifundefined{KOMAClassName}{% if non-KOMA class
  \IfFileExists{parskip.sty}{%
    \usepackage{parskip}
  }{% else
    \setlength{\parindent}{0pt}
    \setlength{\parskip}{6pt plus 2pt minus 1pt}}
}{% if KOMA class
  \KOMAoptions{parskip=half}}
\makeatother
\usepackage{xcolor}
\usepackage[left=2cm, right=2cm, top=2.5cm, bottom=2.5cm]{geometry}
\setlength{\emergencystretch}{3em} % prevent overfull lines
\setcounter{secnumdepth}{5}
% Make \paragraph and \subparagraph free-standing
\ifx\paragraph\undefined\else
  \let\oldparagraph\paragraph
  \renewcommand{\paragraph}[1]{\oldparagraph{#1}\mbox{}}
\fi
\ifx\subparagraph\undefined\else
  \let\oldsubparagraph\subparagraph
  \renewcommand{\subparagraph}[1]{\oldsubparagraph{#1}\mbox{}}
\fi

\usepackage{color}
\usepackage{fancyvrb}
\newcommand{\VerbBar}{|}
\newcommand{\VERB}{\Verb[commandchars=\\\{\}]}
\DefineVerbatimEnvironment{Highlighting}{Verbatim}{commandchars=\\\{\}}
% Add ',fontsize=\small' for more characters per line
\usepackage{framed}
\definecolor{shadecolor}{RGB}{241,243,245}
\newenvironment{Shaded}{\begin{snugshade}}{\end{snugshade}}
\newcommand{\AlertTok}[1]{\textcolor[rgb]{0.68,0.00,0.00}{#1}}
\newcommand{\AnnotationTok}[1]{\textcolor[rgb]{0.37,0.37,0.37}{#1}}
\newcommand{\AttributeTok}[1]{\textcolor[rgb]{0.40,0.45,0.13}{#1}}
\newcommand{\BaseNTok}[1]{\textcolor[rgb]{0.68,0.00,0.00}{#1}}
\newcommand{\BuiltInTok}[1]{\textcolor[rgb]{0.00,0.23,0.31}{#1}}
\newcommand{\CharTok}[1]{\textcolor[rgb]{0.13,0.47,0.30}{#1}}
\newcommand{\CommentTok}[1]{\textcolor[rgb]{0.37,0.37,0.37}{#1}}
\newcommand{\CommentVarTok}[1]{\textcolor[rgb]{0.37,0.37,0.37}{\textit{#1}}}
\newcommand{\ConstantTok}[1]{\textcolor[rgb]{0.56,0.35,0.01}{#1}}
\newcommand{\ControlFlowTok}[1]{\textcolor[rgb]{0.00,0.23,0.31}{#1}}
\newcommand{\DataTypeTok}[1]{\textcolor[rgb]{0.68,0.00,0.00}{#1}}
\newcommand{\DecValTok}[1]{\textcolor[rgb]{0.68,0.00,0.00}{#1}}
\newcommand{\DocumentationTok}[1]{\textcolor[rgb]{0.37,0.37,0.37}{\textit{#1}}}
\newcommand{\ErrorTok}[1]{\textcolor[rgb]{0.68,0.00,0.00}{#1}}
\newcommand{\ExtensionTok}[1]{\textcolor[rgb]{0.00,0.23,0.31}{#1}}
\newcommand{\FloatTok}[1]{\textcolor[rgb]{0.68,0.00,0.00}{#1}}
\newcommand{\FunctionTok}[1]{\textcolor[rgb]{0.28,0.35,0.67}{#1}}
\newcommand{\ImportTok}[1]{\textcolor[rgb]{0.00,0.46,0.62}{#1}}
\newcommand{\InformationTok}[1]{\textcolor[rgb]{0.37,0.37,0.37}{#1}}
\newcommand{\KeywordTok}[1]{\textcolor[rgb]{0.00,0.23,0.31}{#1}}
\newcommand{\NormalTok}[1]{\textcolor[rgb]{0.00,0.23,0.31}{#1}}
\newcommand{\OperatorTok}[1]{\textcolor[rgb]{0.37,0.37,0.37}{#1}}
\newcommand{\OtherTok}[1]{\textcolor[rgb]{0.00,0.23,0.31}{#1}}
\newcommand{\PreprocessorTok}[1]{\textcolor[rgb]{0.68,0.00,0.00}{#1}}
\newcommand{\RegionMarkerTok}[1]{\textcolor[rgb]{0.00,0.23,0.31}{#1}}
\newcommand{\SpecialCharTok}[1]{\textcolor[rgb]{0.37,0.37,0.37}{#1}}
\newcommand{\SpecialStringTok}[1]{\textcolor[rgb]{0.13,0.47,0.30}{#1}}
\newcommand{\StringTok}[1]{\textcolor[rgb]{0.13,0.47,0.30}{#1}}
\newcommand{\VariableTok}[1]{\textcolor[rgb]{0.07,0.07,0.07}{#1}}
\newcommand{\VerbatimStringTok}[1]{\textcolor[rgb]{0.13,0.47,0.30}{#1}}
\newcommand{\WarningTok}[1]{\textcolor[rgb]{0.37,0.37,0.37}{\textit{#1}}}

\providecommand{\tightlist}{%
  \setlength{\itemsep}{0pt}\setlength{\parskip}{0pt}}\usepackage{longtable,booktabs,array}
\usepackage{calc} % for calculating minipage widths
% Correct order of tables after \paragraph or \subparagraph
\usepackage{etoolbox}
\makeatletter
\patchcmd\longtable{\par}{\if@noskipsec\mbox{}\fi\par}{}{}
\makeatother
% Allow footnotes in longtable head/foot
\IfFileExists{footnotehyper.sty}{\usepackage{footnotehyper}}{\usepackage{footnote}}
\makesavenoteenv{longtable}
\usepackage{graphicx}
\makeatletter
\def\maxwidth{\ifdim\Gin@nat@width>\linewidth\linewidth\else\Gin@nat@width\fi}
\def\maxheight{\ifdim\Gin@nat@height>\textheight\textheight\else\Gin@nat@height\fi}
\makeatother
% Scale images if necessary, so that they will not overflow the page
% margins by default, and it is still possible to overwrite the defaults
% using explicit options in \includegraphics[width, height, ...]{}
\setkeys{Gin}{width=\maxwidth,height=\maxheight,keepaspectratio}
% Set default figure placement to htbp
\makeatletter
\def\fps@figure{htbp}
\makeatother
\newlength{\cslhangindent}
\setlength{\cslhangindent}{1.5em}
\newlength{\csllabelwidth}
\setlength{\csllabelwidth}{3em}
\newlength{\cslentryspacingunit} % times entry-spacing
\setlength{\cslentryspacingunit}{\parskip}
\newenvironment{CSLReferences}[2] % #1 hanging-ident, #2 entry spacing
 {% don't indent paragraphs
  \setlength{\parindent}{0pt}
  % turn on hanging indent if param 1 is 1
  \ifodd #1
  \let\oldpar\par
  \def\par{\hangindent=\cslhangindent\oldpar}
  \fi
  % set entry spacing
  \setlength{\parskip}{#2\cslentryspacingunit}
 }%
 {}
\usepackage{calc}
\newcommand{\CSLBlock}[1]{#1\hfill\break}
\newcommand{\CSLLeftMargin}[1]{\parbox[t]{\csllabelwidth}{#1}}
\newcommand{\CSLRightInline}[1]{\parbox[t]{\linewidth - \csllabelwidth}{#1}\break}
\newcommand{\CSLIndent}[1]{\hspace{\cslhangindent}#1}

\usepackage{booktabs}
\usepackage{multirow}
\usepackage{array,graphicx}
\usepackage{wrapfig}
\usepackage[normalem]{ulem}
\usepackage{lscape}
\usepackage{rotating}
\useunder{\uline}{\ul}{}
\usepackage{float}
\floatplacement{figure}{H}
\renewcommand{\chaptername}{Capitolul}
\renewcommand{\contentsname}{Cuprins}
\renewcommand\listfigurename{Lista figurilor}
\renewcommand\listtablename{Lista tabelelor}
\renewcommand\figurename{Figura}
\renewcommand\tablename{Tabel}
\renewcommand\bibname{Bibliografie}
\newcommand*\rot{\rotatebox{90}}
\usepackage{tocbibind}
\makeatletter
\makeatother
\makeatletter
\@ifpackageloaded{bookmark}{}{\usepackage{bookmark}}
\makeatother
\makeatletter
\@ifpackageloaded{caption}{}{\usepackage{caption}}
\AtBeginDocument{%
\ifdefined\contentsname
  \renewcommand*\contentsname{Table of contents}
\else
  \newcommand\contentsname{Table of contents}
\fi
\ifdefined\listfigurename
  \renewcommand*\listfigurename{Lista figurilor}
\else
  \newcommand\listfigurename{Lista figurilor}
\fi
\ifdefined\listtablename
  \renewcommand*\listtablename{Lista tabelelor}
\else
  \newcommand\listtablename{Lista tabelelor}
\fi
\ifdefined\figurename
  \renewcommand*\figurename{Figura}
\else
  \newcommand\figurename{Figura}
\fi
\ifdefined\tablename
  \renewcommand*\tablename{Tabel}
\else
  \newcommand\tablename{Tabel}
\fi
}
\@ifpackageloaded{float}{}{\usepackage{float}}
\floatstyle{ruled}
\@ifundefined{c@chapter}{\newfloat{codelisting}{h}{lop}}{\newfloat{codelisting}{h}{lop}[chapter]}
\floatname{codelisting}{Listing}
\newcommand*\listoflistings{\listof{codelisting}{List of Listings}}
\makeatother
\makeatletter
\@ifpackageloaded{caption}{}{\usepackage{caption}}
\@ifpackageloaded{subcaption}{}{\usepackage{subcaption}}
\makeatother
\makeatletter
\@ifpackageloaded{tcolorbox}{}{\usepackage[many]{tcolorbox}}
\makeatother
\makeatletter
\@ifundefined{shadecolor}{\definecolor{shadecolor}{rgb}{.97, .97, .97}}
\makeatother
\makeatletter
\makeatother
\ifLuaTeX
  \usepackage{selnolig}  % disable illegal ligatures
\fi
\IfFileExists{bookmark.sty}{\usepackage{bookmark}}{\usepackage{hyperref}}
\IfFileExists{xurl.sty}{\usepackage{xurl}}{} % add URL line breaks if available
\urlstyle{same} % disable monospaced font for URLs
\hypersetup{
  pdftitle={Introducere în analiza datelor economico-financiare},
  pdfauthor={Ciprian Alexandru-Caragea},
  hidelinks,
  pdfcreator={LaTeX via pandoc}}

\title{Introducere în analiza datelor economico-financiare}
\usepackage{etoolbox}
\makeatletter
\providecommand{\subtitle}[1]{% add subtitle to \maketitle
  \apptocmd{\@title}{\par {\large #1 \par}}{}{}
}
\makeatother
\subtitle{Aplicații și exemple rezolvate prin: R, Python, Excel,
PowerBI}
\author{Ciprian Alexandru-Caragea}
\date{9/8/24}

\begin{document}
\frontmatter
\maketitle
\ifdefined\Shaded\renewenvironment{Shaded}{\begin{tcolorbox}[interior hidden, enhanced, borderline west={3pt}{0pt}{shadecolor}, boxrule=0pt, breakable, sharp corners, frame hidden]}{\end{tcolorbox}}\fi

\renewcommand*\contentsname{Cuprins}
{
\setcounter{tocdepth}{1}
\tableofcontents
}
\listoffigures
\listoftables
\mainmatter
\bookmarksetup{startatroot}

\hypertarget{despre-autor}{%
\chapter*{Despre autor}\label{despre-autor}}
\addcontentsline{toc}{chapter}{Despre autor}

\markboth{Despre autor}{Despre autor}

\setcounter{page}{3}

\textbf{Ciprian Alexandru-Caragea} este conferenţiar universitar la
Facultatea de Management Financiar, Universitatea Ecologică din
București și Analist de Date la diverse instituții internaționale.\\

\begin{wrapfigure}{r}{0.3\textwidth}
  \begin{center}
    \includegraphics[width=0.3\textwidth]{images/Ciprian_DGINS2018.jpg}
  \end{center}
\end{wrapfigure}

Titlul de doctor în Economie l-a obţinut sub egida Academiei Române,
Institutul de Economie Naţională.\\
A participat la un program de studii postdoctorale în care a implementat
utilizarea software-ului R ca instrument de analiză a evoluției
indicilor bursieri.\\
Activitatea sa didactică se concentrează, în principal, în domeniul
burselor de valori, prin cursuri și seminarii la programele de licență
și masterat (Piețe de capital, Managementul Portofoliului, Piețe
internaționale de capital).\\
A participat la diverse proiecte de cercetare, workshop-uri, conferințe
naționale și internaționale. Activitatea de cercetare a fost pusă în
valoare prin publicarea studiilor în reviste din țară și din Europa,
precum și în baze de date internaționale recunoscute (RePEC, DOAJ,
EBSCO).\\
În prezent, în cadrul Institului Național de Statistică, participă ca
expert în proiecte BigData și utilizează software-ul de analiză
statistică R pentru Data cleaning, Data Matching, Web Scraping, analize
de date și vizualizare, Data mining, Data integration, data processing,
data validation, dar și utilizarea datelor din sursele administrative
pentru realizarea de statistici oficiale.

\href{http://www.researcherid.com/rid/V-2168-2017}{ResearcherID:
V-2168-2017}\\
https://orcid.org/0000-0001-8215-6671

\bookmarksetup{startatroot}

\hypertarget{introducere}{%
\chapter*{Introducere}\label{introducere}}
\addcontentsline{toc}{chapter}{Introducere}

\markboth{Introducere}{Introducere}

\begin{quote}
``\ldots acum nu mai e nimic nou de descoperit;\\
tot ce rămâne e doar măsurătoarea din ce în ce mai precisă''

--- Lord Kelvin (1894)
\end{quote}

Cartea tipărită merită răsfoită. Trăim în vremea în care internetul
facilitează comunicarea globală, informația fiind disponibilă oricând și
oricum. Toată lumea, de la oameni de știință și până la copii de vârstă
școlară primesc și oferă informații și propagă idei pe calea
internetului. Tirajele publicațiilor, cărților și manualelor tipărite
sunt în scădere în întreaga lume, în timp ce postările online captează
atenția omenirii.\\
Obiectivul principal al cărții pe care o propun este de a fi un ghid
cuprinzător, în termeni de concepte și tehnici, reprezentativ și, mai
ales, practic, în ceea ce privește utilizarea instrumentelor software de
analiză statistică, R fiind principalul software utilizat pentru
aplicațiile propuse. Ca abordare generală, cartea prezintă principalele
concepte utilizate în statistică, cu exemple și explicații descriptive.
Exemplele din viața economică - cele mai multe dintre ele bazate pe date
statistice reale - problemele rezolvate, dar și cele propuse, acoperă o
arie cuprinzătoare de tematici, cititorul având șansa de a fi introdus
în sfera aplicativă a conceptelor teoretice parcurse.\\
Cartea este destinată tuturor celor care doresc să înțeleagă, prin
mijloace științifice, fenomenele economice și sociale, sub aspectul
măsurării cantitative și din perspectiva determinării cauzale. Deși se
adresează, în principal, studenților care se pregătesc să devină
specialiști în științele economice, lucrarea este utilă și celor care
își propun să cunoască un domeniu atât de frumos și de captivant. Tocmai
nevoia de informații, din ce în ce mai complexe, dar și posibilitățile
de calcul avansat cu ajutorul soft-urilor tot mai performante, au condus
la crearea unui bazin imens de date care pot fi cu ușurință exploatate
pe baza analizei statistice. Poate că acesta este și motivul pentru care
statistica rămâne o disciplină percepută ca fiind adesea prea
matematizată, destinată specialiștilor. Pentru mulți cititori, mai ales
dintre cei care nu au o formare bazată pe un aparat matematic, studiul
fenomenelor economice prin metode statistice și matematice, presupune un
efort deosebit. Din acest motiv, am încercat să tratez aspectele
teoretice, dar și problemele cu aplicație practică din sfera economică,
într-o manieră simplă, accesibilă. Așadar, lucrarea are menirea de a
facilita înțelegerea conceptelor fundamentale cu care operează
statistica, utilizarea adecvată a metodelor de analiză statistică,
precum și interpretarea corectă a rezultatelor, în vederea cunoașterii
modului de manifestare a fenomenelor.

\begin{flushright}
Nicoleta Caragea \\
Septembrie, 2018
\end{flushright}

\newpage

\hypertarget{quarto}{%
\section*{Quarto}\label{quarto}}
\addcontentsline{toc}{section}{Quarto}

\markright{Quarto}

Această carte a fost editată cu ajutorul pachetului R \textbf{bookdown}
(\protect\hyperlink{ref-xie2015}{\textbf{xie2015?}}).\\
Cartea are la bază manualul \emph{Statistică - concepte și metode de
analiză a
datelor}(\protect\hyperlink{ref-caragea2015}{\textbf{caragea2015?}}).

Pachetul R \textbf{bookdown} este integrat R Markdown
(http://rmarkdown.rstudio.com). Documentele elaborate pe baza acestui
tip de instrumentar de editare sunt pe deplin reproductibile și dau
posibilitatea creării unor formate de ieșire diverse
(PDF/HTML/Word/\ldots). Informații suplimentare referitoare la
utilizarea pachetului \textbf{bookdown} se pot găsi la adresa:
https://bookdown.org.

\includegraphics[width=0.2\textwidth]{images/logo.png}

\hypertarget{informaux21bii-despre-software}{%
\section*{Informații despre
software}\label{informaux21bii-despre-software}}
\addcontentsline{toc}{section}{Informații despre software}

\markright{Informații despre software}

Software-ul \textsf{R} a devenit în prezent unul dintre cele mai
utilizate instrumente de analiză statistică, fiind utilizat în
statisticile oficiale, în mediile universitare și de cercetare
academică, dar și în mediul de afaceri. Acest manual este destinat
tuturor celor care doresc să învețe statistica, fiind un material
introductiv de studiu, care prezintă un spectru larg de exemple,
prezentări grafice și analiză a datelor, dezvoltate cu ajutorul
\textsf{R}.\\
Aplicațiile din această carte utilizează \textsf{R}, ceea ce înseamnă că
pentru reproducerea acestora va fi nevoie de instalarea \textsf{R} pe
calculatorul pe care lucrați.\\
\textsf{R} este un sistem pentru analize statistice și reprezentare
grafică creat de către Ross Ihaka și Robert Gentleman, profesori de
statistică la Universitatea Auckland din Noua
Zeelandă\footnote{Ihaka R. \& Gentleman R. 1996. R: a language for data analysis and graphics. {\it Journal of Computational and Graphical Statistics} 5: 299--314.}.\\
\index{limbajul S}\textsf{R} este considerat un dialect al limbajului
\textsf{S} creat de AT\&T Bell Laboratories. \textsf{S} este disponibil
sub forma software-ului S-PLUS, comercializat de compania Insightful.
Există diferențe importante între cele două limbaje, \textsf{R} și
\textsf{S}: acestea sunt documentate de către Ihaka \& Gentleman (1996)
sau se regăsesc în
R-FAQ\footnote{\href{http://cran.r-project.org/doc/FAQ/R-FAQ.html\#What-are-the-differences-between-R-and-S_003f}{R-FAQ}}.\\
Astfel, numele limbajului R provine de la inițiala prenumelui
creatorilor, dar este totodată și un omagiu adus limbajului
\textsf{S}.\\
În primul rând, \textsf{R} este open-source, fiind distribuit în mod
gratuit sub licență
\textit{GNU - General Public Licence}\footnote{\href{http://www.gnu.org/}{GNU}};
dezvoltarea și distribuirea sunt în grija câtorva profesori și
statisticieni, afiliați companiilor și universităților, cunoscuți sub
denumirea generică de \textit{R Development Core Team}.\\
Conform filosofiei
\textit{GNU}\footnote{\href{http://www.gnu.org/philosophy/free-sw.ro.html\#exportcontrol}{GNU Philosophy}},
software-ul open-source este caracterizat de libertatea acordată
utilizatorilor săi de a-l utiliza, copia, distribui, studia, modifica și
îmbunătăți. Mai exact, este vorba de patru forme de libertate acordate
utilizatorilor(\protect\hyperlink{ref-dusa2015}{\textbf{dusa2015?}}):

\begin{itemize}
\item Libertatea de a utiliza programul, în orice scop (libertatea 0);
\item Libertatea de a studia modul de funcționare a programului, și de a-l adapta nevoilor proprii (libertatea 1). Accesul la codul-sursă este o precondiție pentru aceasta;
\item Libertatea de a redistribui copii, în scopul ajutorării aproapelui tău (libertatea 2);
\item Libertatea de a îmbunătăți programul, și de a pune îmbunătățirile la dispoziția publicului, în folosul întregii societăți (libertatea 3). Accesul la codul-sursă este o precondiție pentru aceasta.
\end{itemize}

Faptul că este gratuit atrage automat avantajul competitiv în fața altor
software-uri de analiză statistică, precum Stata, SAS și SPSS. Astfel,
costurile alocate licenței de software dispar. \textsf{R} este denumit
de către Norman Nie, unul dintre fondatorii SPSS și CEO al Revolution
Analytics, ``cel mai puternic și flexibil limbaj de programare
statistică din lume'' (în engleză
\textit{"the most powerful and flexible statistical programming language in the world"}).\footnote{\href{http://blog.revolutionanalytics.com/2010/10/r-is-hot.html}{Smith, D., 2010,"R is Hot", Revolution Analytics}}
Dovadă a succesului pe care \textsf{R} îl are în știința datelor, s-au
dezvoltat medii de integrare a acestuia în SAS și chiar SPSS. Este vorba
despre modulul
SAS/IML\footnote{\href{http://www.sas.com/en_us/software/analytics/iml.html\#close}{SAS/IML Module}},
care integrează limbajul \textsf{R} în SAS, și despre
\textit{translate2R}, un serviciu de translatare a codului SPSS direct
în \textsf{R} dezvoltat de compania
\textit{eoda}\footnote{\href{http://www.eoda.de/en/translate2R.html}{translate2R - eoda}}.
\textsf{R} are susținerea comunității științifice, dar și a multor
companii internaționale. Dintre acestea, menționăm: Google, Facebook,
Mozilla, Twitter, The New York Times, The Economist, NewScientist,
Lloyd's, Bing, Johnson\&Johnson, Pfizer, Shell, Bank of America,
Ford.\footnote{\href{http://www.revolutionanalytics.com/what-is-open-source-r/companies-using-r.php}{Revolution Analytics, "Companies Using R"}}
\index{open-source}\textsf{R} este susținut și de mediul academic.
Marile universități din lume sprijină \textsf{R}, la fel cum sprijină și
alte inițiative sau software-uri open-source, precum sistemul de operare
Linux sau sistemul de preparare a documentelor \LaTeX.

\bookmarksetup{startatroot}

\hypertarget{cap1}{%
\chapter{Introducere în analiza datelor}\label{cap1}}

\hypertarget{importanux21ba-datelor-uxeen-luare-deciziilor}{%
\section{Importanța datelor în luare
deciziilor}\label{importanux21ba-datelor-uxeen-luare-deciziilor}}

text

\hypertarget{identificarea-seturilor-de-date-relevante}{%
\section{Identificarea seturilor de date
relevante}\label{identificarea-seturilor-de-date-relevante}}

text

\hypertarget{surse-de-date-publice-ux219i-private-inclusiv-open-data}{%
\subsection{Surse de date publice și private, inclusiv Open
Data}\label{surse-de-date-publice-ux219i-private-inclusiv-open-data}}

text

\hypertarget{evaluarea-calitux103ux21bii-ux219i-relevanux21bei-datelor}{%
\subsection{Evaluarea calității și relevanței
datelor}\label{evaluarea-calitux103ux21bii-ux219i-relevanux21bei-datelor}}

text

\hypertarget{utilizarea-platformelor-de-date}{%
\subsection{Utilizarea platformelor de
date}\label{utilizarea-platformelor-de-date}}

• Modalități de acces, descărcare și integrare a datelor • Prezentarea
unor platforme comune: Eurostat, INS, UN data, OECD Data

\hypertarget{concepte-de-bazux103-legate-de-formatarea-datelor}{%
\section{Concepte de bază legate de formatarea
datelor}\label{concepte-de-bazux103-legate-de-formatarea-datelor}}

(formatul și tipurile de date)

\hypertarget{tipuri-de-surse-de-date-primare-ux219i-secundare}{%
\subsection{Tipuri de surse de date (primare și
secundare)}\label{tipuri-de-surse-de-date-primare-ux219i-secundare}}

text

\hypertarget{tipuri-de-fiux219iere-csv-xml-json-ux219i-utilizarea-lor}{%
\subsection{Tipuri de fișiere (CSV, XML, JSON) și utilizarea
lor}\label{tipuri-de-fiux219iere-csv-xml-json-ux219i-utilizarea-lor}}

text

\hypertarget{importanux21ba-ux219i-utilizarea-metadatelor}{%
\subsection{Importanța și utilizarea
metadatelor}\label{importanux21ba-ux219i-utilizarea-metadatelor}}

text

\hypertarget{instrumente-ux219i-software-pentru-analiza-datelor}{%
\section{Instrumente și Software pentru Analiza
Datelor}\label{instrumente-ux219i-software-pentru-analiza-datelor}}

\hypertarget{utilizarea-excel-pentru-analize-financiare}{%
\subsection{Utilizarea Excel pentru analize
financiare}\label{utilizarea-excel-pentru-analize-financiare}}

text

\hypertarget{utilizarea-power-bi-pentru-analize-financiare}{%
\subsection{Utilizarea Power BI pentru analize
financiare}\label{utilizarea-power-bi-pentru-analize-financiare}}

text

\hypertarget{introducere-uxeen-r-ux219i-python-pentru-analize-statistice}{%
\subsection{Introducere în R și Python pentru analize
statistice}\label{introducere-uxeen-r-ux219i-python-pentru-analize-statistice}}

text

\bookmarksetup{startatroot}

\hypertarget{cap2}{%
\chapter{Introducere în analiza statistică}\label{cap2}}

\hypertarget{importanux21ba-datelor-uxeen-luare-deciziilor-1}{%
\section{Importanța datelor în luare
deciziilor}\label{importanux21ba-datelor-uxeen-luare-deciziilor-1}}

text

\hypertarget{identificarea-seturilor-de-date-relevante-1}{%
\section{Identificarea seturilor de date
relevante}\label{identificarea-seturilor-de-date-relevante-1}}

text

\hypertarget{surse-de-date-publice-ux219i-private-inclusiv-open-data-1}{%
\subsection{Surse de date publice și private, inclusiv Open
Data}\label{surse-de-date-publice-ux219i-private-inclusiv-open-data-1}}

text

\hypertarget{evaluarea-calitux103ux21bii-ux219i-relevanux21bei-datelor-1}{%
\subsection{Evaluarea calității și relevanței
datelor}\label{evaluarea-calitux103ux21bii-ux219i-relevanux21bei-datelor-1}}

text

\hypertarget{utilizarea-platformelor-de-date-1}{%
\subsection{Utilizarea platformelor de
date}\label{utilizarea-platformelor-de-date-1}}

• Modalități de acces, descărcare și integrare a datelor • Prezentarea
unor platforme comune: Eurostat, INS, UN data, OECD Data

\hypertarget{concepte-de-bazux103-legate-de-formatarea-datelor-1}{%
\section{Concepte de bază legate de formatarea
datelor}\label{concepte-de-bazux103-legate-de-formatarea-datelor-1}}

(formatul și tipurile de date)

\hypertarget{tipuri-de-surse-de-date-primare-ux219i-secundare-1}{%
\subsection{Tipuri de surse de date (primare și
secundare)}\label{tipuri-de-surse-de-date-primare-ux219i-secundare-1}}

text

\hypertarget{tipuri-de-fiux219iere-csv-xml-json-ux219i-utilizarea-lor-1}{%
\subsection{Tipuri de fișiere (CSV, XML, JSON) și utilizarea
lor}\label{tipuri-de-fiux219iere-csv-xml-json-ux219i-utilizarea-lor-1}}

text

\hypertarget{importanux21ba-ux219i-utilizarea-metadatelor-1}{%
\subsection{Importanța și utilizarea
metadatelor}\label{importanux21ba-ux219i-utilizarea-metadatelor-1}}

text

\hypertarget{instrumente-ux219i-software-pentru-analiza-datelor-1}{%
\section{Instrumente și Software pentru Analiza
Datelor}\label{instrumente-ux219i-software-pentru-analiza-datelor-1}}

\hypertarget{utilizarea-excel-pentru-analize-financiare-1}{%
\subsection{Utilizarea Excel pentru analize
financiare}\label{utilizarea-excel-pentru-analize-financiare-1}}

text

\hypertarget{utilizarea-power-bi-pentru-analize-financiare-1}{%
\subsection{Utilizarea Power BI pentru analize
financiare}\label{utilizarea-power-bi-pentru-analize-financiare-1}}

text

\hypertarget{introducere-uxeen-r-ux219i-python-pentru-analize-statistice-1}{%
\subsection{Introducere în R și Python pentru analize
statistice}\label{introducere-uxeen-r-ux219i-python-pentru-analize-statistice-1}}

text

\bookmarksetup{startatroot}

\hypertarget{cap3}{%
\chapter{Analiza datelor economico-financiare}\label{cap3}}

\hypertarget{importanux21ba-datelor-uxeen-luare-deciziilor-2}{%
\section{Importanța datelor în luare
deciziilor}\label{importanux21ba-datelor-uxeen-luare-deciziilor-2}}

text

\hypertarget{identificarea-seturilor-de-date-relevante-2}{%
\section{Identificarea seturilor de date
relevante}\label{identificarea-seturilor-de-date-relevante-2}}

text

\hypertarget{surse-de-date-publice-ux219i-private-inclusiv-open-data-2}{%
\subsection{Surse de date publice și private, inclusiv Open
Data}\label{surse-de-date-publice-ux219i-private-inclusiv-open-data-2}}

text

\hypertarget{evaluarea-calitux103ux21bii-ux219i-relevanux21bei-datelor-2}{%
\subsection{Evaluarea calității și relevanței
datelor}\label{evaluarea-calitux103ux21bii-ux219i-relevanux21bei-datelor-2}}

text

\hypertarget{utilizarea-platformelor-de-date-2}{%
\subsection{Utilizarea platformelor de
date}\label{utilizarea-platformelor-de-date-2}}

• Modalități de acces, descărcare și integrare a datelor • Prezentarea
unor platforme comune: Eurostat, INS, UN data, OECD Data

\hypertarget{concepte-de-bazux103-legate-de-formatarea-datelor-2}{%
\section{Concepte de bază legate de formatarea
datelor}\label{concepte-de-bazux103-legate-de-formatarea-datelor-2}}

(formatul și tipurile de date)

\hypertarget{tipuri-de-surse-de-date-primare-ux219i-secundare-2}{%
\subsection{Tipuri de surse de date (primare și
secundare)}\label{tipuri-de-surse-de-date-primare-ux219i-secundare-2}}

text

\hypertarget{tipuri-de-fiux219iere-csv-xml-json-ux219i-utilizarea-lor-2}{%
\subsection{Tipuri de fișiere (CSV, XML, JSON) și utilizarea
lor}\label{tipuri-de-fiux219iere-csv-xml-json-ux219i-utilizarea-lor-2}}

text

\hypertarget{importanux21ba-ux219i-utilizarea-metadatelor-2}{%
\subsection{Importanța și utilizarea
metadatelor}\label{importanux21ba-ux219i-utilizarea-metadatelor-2}}

text

\hypertarget{instrumente-ux219i-software-pentru-analiza-datelor-2}{%
\section{Instrumente și Software pentru Analiza
Datelor}\label{instrumente-ux219i-software-pentru-analiza-datelor-2}}

\hypertarget{utilizarea-excel-pentru-analize-financiare-2}{%
\subsection{Utilizarea Excel pentru analize
financiare}\label{utilizarea-excel-pentru-analize-financiare-2}}

text

\hypertarget{utilizarea-power-bi-pentru-analize-financiare-2}{%
\subsection{Utilizarea Power BI pentru analize
financiare}\label{utilizarea-power-bi-pentru-analize-financiare-2}}

text

\hypertarget{introducere-uxeen-r-ux219i-python-pentru-analize-statistice-2}{%
\subsection{Introducere în R și Python pentru analize
statistice}\label{introducere-uxeen-r-ux219i-python-pentru-analize-statistice-2}}

text

\bookmarksetup{startatroot}

\hypertarget{cap4}{%
\chapter{Analiza bugetară}\label{cap4}}

\hypertarget{importanux21ba-datelor-uxeen-luare-deciziilor-3}{%
\section{Importanța datelor în luare
deciziilor}\label{importanux21ba-datelor-uxeen-luare-deciziilor-3}}

text

\hypertarget{identificarea-seturilor-de-date-relevante-3}{%
\section{Identificarea seturilor de date
relevante}\label{identificarea-seturilor-de-date-relevante-3}}

text

\hypertarget{surse-de-date-publice-ux219i-private-inclusiv-open-data-3}{%
\subsection{Surse de date publice și private, inclusiv Open
Data}\label{surse-de-date-publice-ux219i-private-inclusiv-open-data-3}}

text

\hypertarget{evaluarea-calitux103ux21bii-ux219i-relevanux21bei-datelor-3}{%
\subsection{Evaluarea calității și relevanței
datelor}\label{evaluarea-calitux103ux21bii-ux219i-relevanux21bei-datelor-3}}

text

\hypertarget{utilizarea-platformelor-de-date-3}{%
\subsection{Utilizarea platformelor de
date}\label{utilizarea-platformelor-de-date-3}}

• Modalități de acces, descărcare și integrare a datelor • Prezentarea
unor platforme comune: Eurostat, INS, UN data, OECD Data

\hypertarget{concepte-de-bazux103-legate-de-formatarea-datelor-3}{%
\section{Concepte de bază legate de formatarea
datelor}\label{concepte-de-bazux103-legate-de-formatarea-datelor-3}}

(formatul și tipurile de date)

\hypertarget{tipuri-de-surse-de-date-primare-ux219i-secundare-3}{%
\subsection{Tipuri de surse de date (primare și
secundare)}\label{tipuri-de-surse-de-date-primare-ux219i-secundare-3}}

text

\hypertarget{tipuri-de-fiux219iere-csv-xml-json-ux219i-utilizarea-lor-3}{%
\subsection{Tipuri de fișiere (CSV, XML, JSON) și utilizarea
lor}\label{tipuri-de-fiux219iere-csv-xml-json-ux219i-utilizarea-lor-3}}

text

\hypertarget{importanux21ba-ux219i-utilizarea-metadatelor-3}{%
\subsection{Importanța și utilizarea
metadatelor}\label{importanux21ba-ux219i-utilizarea-metadatelor-3}}

text

\hypertarget{instrumente-ux219i-software-pentru-analiza-datelor-3}{%
\section{Instrumente și Software pentru Analiza
Datelor}\label{instrumente-ux219i-software-pentru-analiza-datelor-3}}

\hypertarget{utilizarea-excel-pentru-analize-financiare-3}{%
\subsection{Utilizarea Excel pentru analize
financiare}\label{utilizarea-excel-pentru-analize-financiare-3}}

text

\hypertarget{utilizarea-power-bi-pentru-analize-financiare-3}{%
\subsection{Utilizarea Power BI pentru analize
financiare}\label{utilizarea-power-bi-pentru-analize-financiare-3}}

text

\hypertarget{introducere-uxeen-r-ux219i-python-pentru-analize-statistice-3}{%
\subsection{Introducere în R și Python pentru analize
statistice}\label{introducere-uxeen-r-ux219i-python-pentru-analize-statistice-3}}

text

\bookmarksetup{startatroot}

\hypertarget{cap5}{%
\chapter{Etapele analizei datelor}\label{cap5}}

\hypertarget{importanux21ba-datelor-uxeen-luare-deciziilor-4}{%
\section{Importanța datelor în luare
deciziilor}\label{importanux21ba-datelor-uxeen-luare-deciziilor-4}}

text

\hypertarget{identificarea-seturilor-de-date-relevante-4}{%
\section{Identificarea seturilor de date
relevante}\label{identificarea-seturilor-de-date-relevante-4}}

text

\hypertarget{surse-de-date-publice-ux219i-private-inclusiv-open-data-4}{%
\subsection{Surse de date publice și private, inclusiv Open
Data}\label{surse-de-date-publice-ux219i-private-inclusiv-open-data-4}}

text

\hypertarget{evaluarea-calitux103ux21bii-ux219i-relevanux21bei-datelor-4}{%
\subsection{Evaluarea calității și relevanței
datelor}\label{evaluarea-calitux103ux21bii-ux219i-relevanux21bei-datelor-4}}

text

\hypertarget{utilizarea-platformelor-de-date-4}{%
\subsection{Utilizarea platformelor de
date}\label{utilizarea-platformelor-de-date-4}}

• Modalități de acces, descărcare și integrare a datelor • Prezentarea
unor platforme comune: Eurostat, INS, UN data, OECD Data

\hypertarget{concepte-de-bazux103-legate-de-formatarea-datelor-4}{%
\section{Concepte de bază legate de formatarea
datelor}\label{concepte-de-bazux103-legate-de-formatarea-datelor-4}}

(formatul și tipurile de date)

\hypertarget{tipuri-de-surse-de-date-primare-ux219i-secundare-4}{%
\subsection{Tipuri de surse de date (primare și
secundare)}\label{tipuri-de-surse-de-date-primare-ux219i-secundare-4}}

text

\hypertarget{tipuri-de-fiux219iere-csv-xml-json-ux219i-utilizarea-lor-4}{%
\subsection{Tipuri de fișiere (CSV, XML, JSON) și utilizarea
lor}\label{tipuri-de-fiux219iere-csv-xml-json-ux219i-utilizarea-lor-4}}

text

\hypertarget{importanux21ba-ux219i-utilizarea-metadatelor-4}{%
\subsection{Importanța și utilizarea
metadatelor}\label{importanux21ba-ux219i-utilizarea-metadatelor-4}}

text

\hypertarget{instrumente-ux219i-software-pentru-analiza-datelor-4}{%
\section{Instrumente și Software pentru Analiza
Datelor}\label{instrumente-ux219i-software-pentru-analiza-datelor-4}}

\hypertarget{utilizarea-excel-pentru-analize-financiare-4}{%
\subsection{Utilizarea Excel pentru analize
financiare}\label{utilizarea-excel-pentru-analize-financiare-4}}

text

\hypertarget{utilizarea-power-bi-pentru-analize-financiare-4}{%
\subsection{Utilizarea Power BI pentru analize
financiare}\label{utilizarea-power-bi-pentru-analize-financiare-4}}

text

\hypertarget{introducere-uxeen-r-ux219i-python-pentru-analize-statistice-4}{%
\subsection{Introducere în R și Python pentru analize
statistice}\label{introducere-uxeen-r-ux219i-python-pentru-analize-statistice-4}}

text

\bookmarksetup{startatroot}

\hypertarget{cap6}{%
\chapter{Crearea și partajarea rapoartelor și tablourilor de
bord}\label{cap6}}

\hypertarget{importanux21ba-datelor-uxeen-luare-deciziilor-5}{%
\section{Importanța datelor în luare
deciziilor}\label{importanux21ba-datelor-uxeen-luare-deciziilor-5}}

text

\hypertarget{identificarea-seturilor-de-date-relevante-5}{%
\section{Identificarea seturilor de date
relevante}\label{identificarea-seturilor-de-date-relevante-5}}

text

\hypertarget{surse-de-date-publice-ux219i-private-inclusiv-open-data-5}{%
\subsection{Surse de date publice și private, inclusiv Open
Data}\label{surse-de-date-publice-ux219i-private-inclusiv-open-data-5}}

text

\hypertarget{evaluarea-calitux103ux21bii-ux219i-relevanux21bei-datelor-5}{%
\subsection{Evaluarea calității și relevanței
datelor}\label{evaluarea-calitux103ux21bii-ux219i-relevanux21bei-datelor-5}}

text

\hypertarget{utilizarea-platformelor-de-date-5}{%
\subsection{Utilizarea platformelor de
date}\label{utilizarea-platformelor-de-date-5}}

• Modalități de acces, descărcare și integrare a datelor • Prezentarea
unor platforme comune: Eurostat, INS, UN data, OECD Data

\hypertarget{concepte-de-bazux103-legate-de-formatarea-datelor-5}{%
\section{Concepte de bază legate de formatarea
datelor}\label{concepte-de-bazux103-legate-de-formatarea-datelor-5}}

(formatul și tipurile de date)

\hypertarget{tipuri-de-surse-de-date-primare-ux219i-secundare-5}{%
\subsection{Tipuri de surse de date (primare și
secundare)}\label{tipuri-de-surse-de-date-primare-ux219i-secundare-5}}

text

\hypertarget{tipuri-de-fiux219iere-csv-xml-json-ux219i-utilizarea-lor-5}{%
\subsection{Tipuri de fișiere (CSV, XML, JSON) și utilizarea
lor}\label{tipuri-de-fiux219iere-csv-xml-json-ux219i-utilizarea-lor-5}}

text

\hypertarget{importanux21ba-ux219i-utilizarea-metadatelor-5}{%
\subsection{Importanța și utilizarea
metadatelor}\label{importanux21ba-ux219i-utilizarea-metadatelor-5}}

text

\hypertarget{instrumente-ux219i-software-pentru-analiza-datelor-5}{%
\section{Instrumente și Software pentru Analiza
Datelor}\label{instrumente-ux219i-software-pentru-analiza-datelor-5}}

\hypertarget{utilizarea-excel-pentru-analize-financiare-5}{%
\subsection{Utilizarea Excel pentru analize
financiare}\label{utilizarea-excel-pentru-analize-financiare-5}}

text

\hypertarget{utilizarea-power-bi-pentru-analize-financiare-5}{%
\subsection{Utilizarea Power BI pentru analize
financiare}\label{utilizarea-power-bi-pentru-analize-financiare-5}}

text

\hypertarget{introducere-uxeen-r-ux219i-python-pentru-analize-statistice-5}{%
\subsection{Introducere în R și Python pentru analize
statistice}\label{introducere-uxeen-r-ux219i-python-pentru-analize-statistice-5}}

text

\bookmarksetup{startatroot}

\hypertarget{cap7}{%
\chapter{Calitatea datelor}\label{cap7}}

\hypertarget{importanux21ba-datelor-uxeen-luare-deciziilor-6}{%
\section{Importanța datelor în luare
deciziilor}\label{importanux21ba-datelor-uxeen-luare-deciziilor-6}}

text

\hypertarget{identificarea-seturilor-de-date-relevante-6}{%
\section{Identificarea seturilor de date
relevante}\label{identificarea-seturilor-de-date-relevante-6}}

text

\hypertarget{surse-de-date-publice-ux219i-private-inclusiv-open-data-6}{%
\subsection{Surse de date publice și private, inclusiv Open
Data}\label{surse-de-date-publice-ux219i-private-inclusiv-open-data-6}}

text

\hypertarget{evaluarea-calitux103ux21bii-ux219i-relevanux21bei-datelor-6}{%
\subsection{Evaluarea calității și relevanței
datelor}\label{evaluarea-calitux103ux21bii-ux219i-relevanux21bei-datelor-6}}

text

\hypertarget{utilizarea-platformelor-de-date-6}{%
\subsection{Utilizarea platformelor de
date}\label{utilizarea-platformelor-de-date-6}}

• Modalități de acces, descărcare și integrare a datelor • Prezentarea
unor platforme comune: Eurostat, INS, UN data, OECD Data

\hypertarget{concepte-de-bazux103-legate-de-formatarea-datelor-6}{%
\section{Concepte de bază legate de formatarea
datelor}\label{concepte-de-bazux103-legate-de-formatarea-datelor-6}}

(formatul și tipurile de date)

\hypertarget{tipuri-de-surse-de-date-primare-ux219i-secundare-6}{%
\subsection{Tipuri de surse de date (primare și
secundare)}\label{tipuri-de-surse-de-date-primare-ux219i-secundare-6}}

text

\hypertarget{tipuri-de-fiux219iere-csv-xml-json-ux219i-utilizarea-lor-6}{%
\subsection{Tipuri de fișiere (CSV, XML, JSON) și utilizarea
lor}\label{tipuri-de-fiux219iere-csv-xml-json-ux219i-utilizarea-lor-6}}

text

\hypertarget{importanux21ba-ux219i-utilizarea-metadatelor-6}{%
\subsection{Importanța și utilizarea
metadatelor}\label{importanux21ba-ux219i-utilizarea-metadatelor-6}}

text

\hypertarget{instrumente-ux219i-software-pentru-analiza-datelor-6}{%
\section{Instrumente și Software pentru Analiza
Datelor}\label{instrumente-ux219i-software-pentru-analiza-datelor-6}}

\hypertarget{utilizarea-excel-pentru-analize-financiare-6}{%
\subsection{Utilizarea Excel pentru analize
financiare}\label{utilizarea-excel-pentru-analize-financiare-6}}

text

\hypertarget{utilizarea-power-bi-pentru-analize-financiare-6}{%
\subsection{Utilizarea Power BI pentru analize
financiare}\label{utilizarea-power-bi-pentru-analize-financiare-6}}

text

\hypertarget{introducere-uxeen-r-ux219i-python-pentru-analize-statistice-6}{%
\subsection{Introducere în R și Python pentru analize
statistice}\label{introducere-uxeen-r-ux219i-python-pentru-analize-statistice-6}}

text

\bookmarksetup{startatroot}

\hypertarget{summary}{%
\chapter*{Summary}\label{summary}}
\addcontentsline{toc}{chapter}{Summary}

\markboth{Summary}{Summary}

In summary, this book has no content whatsoever.

\begin{Shaded}
\begin{Highlighting}[]
\DecValTok{1} \SpecialCharTok{+} \DecValTok{1}
\end{Highlighting}
\end{Shaded}

\begin{verbatim}
[1] 2
\end{verbatim}

\bookmarksetup{startatroot}

\hypertarget{bibliografie}{%
\chapter*{Bibliografie}\label{bibliografie}}
\addcontentsline{toc}{chapter}{Bibliografie}

\markboth{Bibliografie}{Bibliografie}

\hypertarget{refs}{}
\begin{CSLReferences}{0}{0}
\end{CSLReferences}


\backmatter

\end{document}
